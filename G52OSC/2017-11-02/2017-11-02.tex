\documentclass{article}
\usepackage[%
    left=0.5in,%
    right=0.5in,%
    top=0.5in,%
    bottom=0.5in,%
]{geometry}%
\usepackage{minitoc}
\usepackage{multicol}
\usepackage{graphicx}
\usepackage{fixltx2e}
\usepackage{hyperref}
\usepackage{hyperref}
    \hypersetup{ colorlinks = true, linkcolor = blue }
\usepackage{blindtext}

\graphicspath{ {./} }

\newcommand{\inlinecode}[2]{\colorbox{lightgray}{\lstinline
[language=#1]$#2$}}
\newcommand{\worddef}[1]{\hyperref[sec:reference]{\textit{#1}}}

\begin{document}
\section{Dynamic Partitioning}
\subsection{Allocating Available Memory: First Fit}
\begin{flushleft}
\textit{First fit algorithm} starts scanning from the start of the linked list until a link is found which represents free space of sufficient size.
\begin{itemize}
	\item If requested space is the \textbf{exact} same size as the “hole”, \textbf{all the space} is allocated
	\item Else, the free link is split into two: \textbf{The first} entry is set to the size requested and marked “used”. \textbf{The second} entry is set to remaining size and marked “free”
\end{itemize}
\end{flushleft}

\subsection{Allocating Available Memory: Next Fit}
\begin{flushleft}
The next fit algorithm maintains a record of where it got to:
\begin{itemize}
	\item  It restarts its search from \textbf{where it stopped} last time
	\item It gives an \textbf{even} chance to all memory to get allocated (first fit concentrates on the start of the list) 
\end{itemize}
However, simulations have shown that next fit actually gives \textbf{worse} performance than first fit!
\end{flushleft}

\subsection{Allocating Available Memory: Best Fit}
\begin{flushleft}
\textbf{The best fit algorithm} always searches the \textbf{entire linked list} to find the smallest hole big enough to satisfy the memory request.
\begin{itemize}
	\item It is \textbf{slower} than first fit
	\item It also results in \textbf{more wasted memory} (memory is more likely to fill up with tiny - useless - holes)
\end{itemize}
\end{flushleft}

\subsection{Allocating Available Memory: Worst Fit}
\begin{flushleft}
\textbf{Tiny holes} are created when best fit split an empty partition.The worst fit algorithm finds the \textbf{largest} available empty partition and splits it.
\begin{itemize}
	\item The left over part will still be large (and potentially more useful)
	\item Simulations have also shown that worst fit is not very good either!
\end{itemize}
\end{flushleft}

\subsection{Allocating Available Memory: Quick Fit and Others}
\begin{flushleft}
Quick fit maintains lists of commonly used sizes
\begin{itemize}
	\item For example a separate list for each of 4K, 8K, 12K, 16K, etc., holes
	\item Odd sizes can either go into the nearest size or into a special separate list 
\end{itemize}
It is much \textbf{faster} to find the required size hole using \textit{quick fit}. Similar to best fit, it has the problem of \textbf{creating many tiny holes}. Finding neighbours for coalescing (combining empty partitions) becomes more \textbf{difficult/time consuming}.
\end{flushleft}

\subsection{Summary}
\begin{itemize}
	\item First fit: allocate first block that is large enough
	\item Next fit: allocate next block that is large enough, i.e. starting from the current location
	\item Best fit: choose block that matches required size closest - O(N) complexity
	\item Worst fit: choose the largest possible block - O(N) complexity
\end{itemize}

\section{Managing available memory}
\subsection{Coalescing}
\begin{flushleft}
\textit{Coalescing} (joining together) takes place when two adjacent entries in the linked list become free. Both neighbours are examined when a block is freed.
\begin{itemize}
	\item If either (or both) are also free
	\item Then the two (or three) entries are \textbf{combined} into one larger block \textbf{by adding up} the sizes
	\item The earlier block in the linked list gives the start point
	\item The separate links are \textbf{deleted} and a \textbf{single link} inserted
\end{itemize}
\end{flushleft}

\subsection{Compacting}
\begin{flushleft}
Even with \textit{coalescing} happening automatically, free blocks may still distributed across memory.\\
\textit{Compacting} can be used to \textbf{join} free and used memory (but is time consuming). Compacting is more \textbf{difficult and time consuming} to implement than coalescing (processes have to be moved).Each process is \textbf{swapped} out and free space coalesced. Process \textbf{swapped back} in at lowest available location
\end{flushleft}

\subsection{Allocation Schemes}
\begin{flushleft}
Different contiguous memory allocation schemes have different advantages/disadvantages:
\begin{itemize}
	\item Mono-programming is \textbf{easy} but does result in \textbf{low} resource utilisation
	\item Fixed partitioning facilitates multi-programming but results in \textbf{internal fragmentation}
	\item Dynamic partitioning facilitates multi-programming, reduces internal fragmentation, but results in \textbf{external fragmentation} (allocation methods, coalescing, and compacting help) 
\end{itemize}
\end{flushleft}

\section{Paging}
\begin{flushleft}
\textit{Paging} uses the principles of fixed partitioning and code re-location to devise a new non-contiguous management scheme:
\begin{itemize}
	\item Memory is \textbf{split into much smaller blocks} and one or multiple blocks are allocated to a process e.g., a 11KB process would take up 3 blocks of 4 KB
	\item These blocks \textbf{do not} have to be contiguous in main memory, but the process still perceives them to be \textbf{contiguous}
	\item Benefits compared to contiguous schemes include: \textit{Internal fragmentation} is reduced to the last “block” only. There is \textbf{no external fragmentation}, since physical blocks are stacked directly onto each other in main memory.
\end{itemize}
A page is a small block of contiguous memory in the \textbf{logical} address space, i.e. as seen by the process.A frame is a small contiguous block in \textbf{physical} memory
\end{flushleft}

\subsection{Relocation}
\begin{flushleft}
Logical address (page number, offset within page) needs to be \textbf{translated into a physical address} (frame number, offset within frame)\\.
Multiple “base registers” will be required:
\begin{itemize}
	\item Each logical page needs a \textbf{separate “base register”} that specifies the start of the associated frame I.e, a set of base registers has to be maintained for each process
	\item The base registers are stored in the \textbf{page table}
\end{itemize}
\bigskip

The page table can be seen as a function, that \textbf{maps} the page number of the logical address onto the frame number of the physical address frameNumber=f(pageNumber). The \textbf{page number} is used as \textbf{index} to the page table that lists the number of the associated frame, i.e. it contains the location of the frame in memory.\\
Every process \textbf{has its own page table} containing its own “base registers”. The operating system maintains a list of free frames.
\end{flushleft}

\pagebreak
\section*{Reference section} \label{sec:reference}
\begin{description}
	\item[placeholder] \hfill \\
\end{description}
\end{document}
