\documentclass{article}

\usepackage[%
    left=0.5in,%
    right=0.5in,%
    top=0.5in,%
    bottom=0.5in,%
]{geometry}%
\usepackage{minitoc}
\usepackage{multicol}
\usepackage{graphicx}
\usepackage{fixltx2e}
\usepackage{listings}
\usepackage{color}
\usepackage{hyperref}
    \hypersetup{ colorlinks = true, linkcolor = blue }
\usepackage{blindtext}
\definecolor{lightgray}{gray}{0.9}
\graphicspath{ {./} }

\newcommand{\inlinecode}[2]{\colorbox{lightgray}{\lstinline
[language=#1]$#2$}}
\newcommand{\worddef}[1]{\hyperref[sec:reference]{\textit{#1}}}

\begin{document}

\tableofcontents

\newpage

\section{Reliance and trust in computers}
Computers in
\begin{itemize}
  \item health services
  \item training and assessment
  \item control of engines and machinery
  \item operational and financial scenarios
  \item genetics
\end{itemize}
In all of these cases computer is making decisions and we humans trust it to make the right one.

\subsection{Expectations of autonomous systems}

``So, rather than verifying the agent never choses a course of action it believes \textbf{will lead} to a bad situation, we would like to verify the agent never chooses a course of action that it believes is \textbf{more likely to reach} a bad situation than its other options''

\section{Why professional ethics}

\begin{flushleft}
In everyday practices we face issues and make decisions. We may be dealing with ethical issues but:
\end{flushleft}
\begin{itemize}
  \item We might not recognise them
  \item We may automatically do the “easy” thing
  \item We may do things we subsequently regret
  \item We may get ourselves and our employer into trouble
  \item We may miss opportunities.
\end{itemize}

\section{Ethics and morality}
\begin{flushleft}
\textbf{Ethics} is a set of \textbf{morally permissible} standards of a group that each member of the group, at his/her rational best, wants every other member to follow, even if their doing so would mean that he/she \textbf{MUST} do the same.

\bigskip

\textbf{Morality} is the set of standards that everyone (\textbf{every rational person at his/her rational best}) wants everyone else to follow, even if their following them means having to do the same.
\end{flushleft}

\subsection{Rational best in ethics reasoning}
\begin{itemize}
  \item Rational best:
  \begin{itemize}
    \item Refers to the mental state of the person who wants the (ethical/moral) rules to be followed.
    \item The person is assumed to be a rational persona whose capacity for reasoning is
not diminished through some form of injury, disease, dugs, grief, fear, etc. 
  \end{itemize}
  \item Practical reasoning
  \begin{itemize}
    \item Used to make decisions.
    \item Often starts with the \worddef{code of conduct} and then takes into account mitigating circumstances.
    \item There is not a set of common axioms/premises and the inference may be disputable.  
  \end{itemize}
\end{itemize}

\subsection{Ethics vs morality}

\begin{flushleft}
Ethical rules are not always moral rules. They can be morally \textbf{neutral}.
\begin{itemize}
  \item If an IT professional delays installing a security patch and a person loses data, moral rules may not have much to say.
\end{itemize}
Breaking ethical rules may not mean breaking moral rules
\begin{itemize}
  \item Publishing salaries of doctors in the USA may be unethical but not immoral
\end{itemize}
\textbf{Ethics} refer to rules provided by an external source, e.g., codes of conduct in workplaces or principles in religions.\\ \textbf{Morals} refer to an individual’s own principles regarding right and wrong.

\end{flushleft}

\section{Properties of moral theories}

\begin{flushleft}
Moral theory defines morality. Our definition of morality may not be acceptable to everybody.
Morality answers to questions: \textbf{How do I know that X is good. Why is X good}
\end{flushleft}

\subsection{Virtue theory}
Virtue ethics are normative ethical theories which emphasize \worddef{virtues} of mind and character. Virtue ethicists discuss the nature and definition of virtues and other related problems. According to Aristotle, the nine most important virtues are: \textbf{wisdom; prudence; justice; fortitude; courage; liberality; magnificence; magnanimity; temperance.}

\subsection{Focusing on the actions and implications}
\begin{itemize}
  \item \textbf{Consequentialism:} Consequences of an action, not the motivation behind the action, makes the action good or bad.
  \item \textbf{Utilitarianism:} Right decision is the one that causes the most happiness*.
  \begin{itemize}
    \item Act utilitarianism: Determine whether or not the action taken maximizes happiness, compared to the other options
    \item Rule utilitarianism: Determine whether or not the action taken complies with the set rules, which are selected so to maximize happiness if \textbf{followed faithfully}. Violation justified only if utility is increased by such violations.
    \begin{itemize}
      \item Do not kill
      \item Do not cause pain
      \item Do not disable
      \item Do not deprive of freedom
      \item Do not deprive of pleasure
      \item Do not deceive
      \item Keep your promise
      \item Do not cheat
      \item Obey the law
      \item Do your duty
    \end{itemize}
  \end{itemize}
  \item \textbf{Deontological} ethical theory: Some rules must be followed, even if they result in a bad end.
  \begin{itemize}
    \item Unlike consequentialism, which judges actions by their results, deontology doesn’t require weighing the costs and benefits of a situation. This avoids subjectivity and uncertainty because you only have to follow set rules.
  \end{itemize}
  \item \textbf{Kantian ethics} based on the view that the only intrinsically good thing is a good will; an action can only be good if its maxim – the principle behind it – is duty to the moral law
  \begin{itemize}
    \item \textbf{Kant}: The only good is good will. 
    \item \textbf{Universal Law of Nature}: Act only according to maxims (principles) that could be adopted as universal laws.
    \item \textbf{End in itself}: Treat humans, both yourself and others, as ends in themselves and never as a means to an end.
  \end{itemize}
  \item \textbf{Social Contract theory}: Individuals establish social contracts with individuals in their vicinity, in order to survive and prosper. The view that persons' moral and/or political obligations \textbf{are dependent upon a contract or agreement among them} to form the society in which they live.
\end{itemize}

\section{Properties of ethics theories}

\begin{flushleft}
All social contract theories of ethics must be fair. We are in search of ethical theorises that have two key qualities:\\
\textbf{Impartiality} Every person is treated equally and no one is given preferential
treatment in the theory.\\
\textbf{Universality} A decision based on the theory should be correct for everyone that has a similar decision to make. 
\end{flushleft}

\section{Ethics of justice - John Rawls}
\begin{flushleft}
Each person should have as extensive set of \textbf{basic freedoms} as possible, as long as it does not \textbf{prevent others} from having the same. Social and economic inequalities are justified only if:
\begin{itemize}
  \item Everyone has a fair chance to obtain the better position 
  \item The worst off will be better off than they might be under an equal distribution
  \item Inequality should not make it harder for those without resources to occupy positions of power
\end{itemize}
\end{flushleft}

\section{Ethics of caring - Nel Noddings}
\begin{flushleft}
We want to be moral to remain in the caring relationship and to enhance the ideal of ourselves as one-caring. Caring is a feeling of engrossment in the needs of another person. To care for a person, the person needs to be \textbf{receptive of the offered care}. While it is possible care-about everyone, it is not possible to carefor everyone.
\end{flushleft}

\pagebreak

\section*{Reference section} \label{sec:reference}
\begin{description}
	\item[code of conduct] \hfill \\ The code of conduct for a group or organization is an agreement on rules of behaviour for the members of that group or organization.
  \item[virtues] \hfill \\ Behaviour showing high moral standards. Good moral quality in a person, or the general quality of being morally good:
\end{description}
\end{document}
