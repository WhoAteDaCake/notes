\documentclass{article}

\usepackage[%
    left=0.5in,%
    right=0.5in,%
    top=0.5in,%
    bottom=0.5in,%
]{geometry}%
\usepackage{minitoc}
\usepackage{multicol}
\usepackage{graphicx}
\usepackage{fixltx2e}
\usepackage{listings}
\usepackage{color}
\usepackage{hyperref}
    \hypersetup{ colorlinks = true, linkcolor = blue }
\usepackage{blindtext}
\definecolor{lightgray}{gray}{0.9}
\graphicspath{ {./} }

\newcommand{\inlinecode}[2]{\colorbox{lightgray}{\lstinline
[language=#1]$#2$}}
\newcommand{\worddef}[1]{\hyperref[sec:reference]{\textit{#1}}}

\begin{document}

\tableofcontents

\newpage

\section{What is critical thinking}

\begin{flushleft}
Ability to think clearly and rationally about what to do or what to believe, being able to engage in reflective and independent thinking 
\end{flushleft}
\begin{itemize}
  \item Understand the logical connections between ideas
  \item Identify, construct and evaluate arguments
  \item Detect inconsistencies and common mistakes in reasoning
  \item Solve problems systematically
  \item Identify the relevance and importance of ideas
  \item Reflect on the justification of one's own beliefs and values (Not being biased)
\end{itemize}

\section{Process of critical thinking}

Analyze
  \begin{itemize}
    \item Identify facts, assumptions, conflicts, flaws, relevance, omissions, consistency, preferences, beliefs, biases, inherited opinions, etc. 
    \item Ask the right questions when analysing: reliable, proven, qualified, quantified, debatable, etc.
    \item \worddef{Scrutinize} the evidence, reasons, arguments, claims
  \end{itemize}
Evaluate
  \begin{itemize}
    \item Tolerate and deal with ambiguity
    \item Stay open-minded to alternatives
    \item Exert intellectual effort, develop reasons based on evidence
  \end{itemize}
Conclude
  \begin{itemize}
    \item Formulate arguments to support claims
    \item Make judgement or suspend it
    \item Change position/stance based on well-founded arguments
    \item Actively think (progress intellectually) and communicate effectively.
  \end{itemize}

\section{Rationality}
\begin{flushleft}
Rationality is the quality or state of being reasonable, \textbf{based on facts or reason}. Rationality implies that one's beliefs are aligned with one's reasons to believe \textbf{or} one's actions are congruent with one's reasons for action.
\end{flushleft}

\section{Logic and reasoning}
Logic is concerned with the principles of correct reasoning:
\begin{itemize}
  \item How do we make conclusions
  \item How do we justify decisions
\end{itemize}

\subsection{Assertions and Arguments}
\textit{Assertion} is a statement, where \textit{proposition} is informational content of any statement or assertion (a statement that can be judged true or false).
\begin{flushleft}
\textit{Argument} is a reason or set of reasons given in support of an idea, action or theory. It offers a series of related statements to support an assertion — to give others good reasons to believe that the assertion is true rather than false.
\end{flushleft}

\subsection{Which propositions makes sense to argue about}
A ‘strong’ proposition has five characteristics:
\begin{itemize}
  \item It is NOT matters of verifiable fact or matters of taste
  \begin{itemize}
    \item If it's a verifiable fact it should be checked rather than argued about
  \end{itemize}
  \item It makes an assertion or urges a course of action in a declarative sentence.
  \begin{itemize}
    \item Should hotels allow dogs in the rooms? \textbf{Not strong}
    \item Hotels should allow dogs in the rooms. \textbf{Strong}
  \end{itemize}
  \item It should not include words that reflect a position on the proposition, i.e., introduce bias
  \begin{itemize}
    \item Example: The inadequate care of parks in the city must be improved
    \item It is necessary to verify that ‘inadequate’ care exists in order to have a proper debate. If the care is inadequate then there is no debate.
  \end{itemize}
  \item It is un-ambiguous, i.e., clear about the idea it states. It should not allow for multiple interpretations.
  \begin{itemize}
    \item For example: High-school students are given scholarship to study ‘data science’. All students with B average would qualify. That is put forward for discussion.
    \item The staff is asking what ‘B average’ means. B average in all subjects or only relevant subjects?
    \item For example, if a student may have A in French and C in mathematics. Would the student be eligible for a technical scholarship?
    \item Could be restated: High-school students are given scholarship to study ‘data science’. All students with B average would qualify. That is put forward for discussion.
  \end{itemize}
  \item It must be singular. One cannot reasonably argue two ideas at once.
\end{itemize}

\section{Turning propositions into arguments}
\begin{flushleft}
Once an arguable proposition is created, we need to identify the \textbf{minor propositions} to support the argument. A minor premise shows the sense of the major proposition – a reason to support the major proposition. For example:
\end{flushleft}
\begin{itemize}
  \item Major: Capital punishment should be abolished.
  \item Minor: Capital punishment is not a deterrent to crime. 
  \item Major: Use of Marijuana by adults in the UK should be legalized.
  \item Minor: Prohibited use of Marijuana is contributing to the illegal activities. 
\end{itemize}
\begin{flushleft}
The minor premise itself may be debatable, in which case one has to argue it individually. When stating the minor premise, one has to \textbf{link it back} to the major premise. When defending the main proposition, one has to consider the points that the opposing side of the argument is putting forward and refute them
\end{flushleft}

\section{Deductive and Inductive Reasoning}
\begin{flushleft}
Deduction and induction used to be differentiated in terms of argument flow, from general to specific and vice-versa. Deduction was seen as \textbf{flowing from a general towards specific statements}.
\begin{itemize}
  \item All men are mortal. PREMISE
  \item Socrates is a man. PREMISE
  \item Therefore, Socrates is mortal. CONCLUSION
\end{itemize}
In a deductive argument, it is \textbf{impossible} for the premises to be \textbf{true} and the conclusion to be \textbf{false}.\\
Induction was seen as \textbf{flow from particular facts to general statements}.
\begin{itemize}
  \item Socrates was Greek. PREMISE
  \item Most Greeks eat fish. PREMISE
  \item Socrates ate fish. CONCLUSION
\end{itemize}
In an inductive argument the premises are supposed to support the conclusion, so that if the premises are true, it is improbable that the conclusion would be false. (Here it's \textbf{assumed} that Socrates is in group of greeks who eat fish)
\end{flushleft}

\section{Informal and formal logic}
\begin{flushleft}
\textbf{Informal logic} is often used to mean critical thinking. \\
\textbf{Formal logic} involves systems that are constructed to carry out proofs, where the languages and rules of reasoning are precisely and carefully defined.
\end{flushleft}

\section{Valid and invalid argument}
\begin{flushleft}
The argument is valid if the \textbf{premises entail their conclusion} (It is logically impossible for its premises to be true and the conclusion to be false.) A valid argument may have \textbf{false premises}. In that case, the implication is valid but the argument is not sound 
\end{flushleft}

\subsection{Fallacties}
\begin{itemize}
  \item Ambiguity: Interpret the meaning of some element of an argument differently.
  \item Circular Arguments: Premises disguise the conclusion. Premises presuppose the truth of the conclusion.
  \item Unwarranted Assumptions: Assuming that some whole, composed of a set of parts, automatically has a certain property shared by each part
  \item Missing Evidence: Weak generalizations from experience that is too small or is biased in some way
  \item Incorrect Causation: Mismatch cause-effect.
  \item Irrelevant Premises: One or more premises not directly relevant to the asserted conclusion
  \item Appeal to Emotion, Authority, Beliefs: Seeking justification
  \item Incorrect Deduction: Argument does not have the structure to be valid
\end{itemize}

\pagebreak
\section*{Reference section} \label{sec:reference}
\begin{description}
	\item[Scrutinize] \hfill \\ examine or inspect closely and thoroughly.
\end{description}
\end{document}
