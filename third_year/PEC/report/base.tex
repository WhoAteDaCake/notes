\documentclass{article}

\usepackage{minitoc}
\usepackage{multicol}
\usepackage{graphicx}
\usepackage{fixltx2e}
\usepackage{listings}
\usepackage{color}
\usepackage{hyperref}
    \hypersetup{ colorlinks = true, linkcolor = blue }
\usepackage{blindtext}
\definecolor{lightgray}{gray}{0.9}
\graphicspath{ {./} }

\usepackage{biblatex}
\bibliography{refs.bib}

\begin{document}

\begin{titlepage}
   \begin{center}
      \Large\textbf{Case study analysis}\\
   \end{center}
   \bigskip
   \textbf{Student name: } Augustinas Jokubauskas\\
   \textbf{Student id: } psyaj1\\
   \textbf{Date of Submission: } \today\\
   \textbf{Word count: } UNKNOWN\\
   \textbf{Declaration: } I confirm that this coursework submission is all my
own work, except where explicitly indicated within
the text.
\end{titlepage}

\newpage

\section{Project title and description}
\begin{flushleft}
Making climate social (MCS) is a project focused on analysing climate change topics across different social media platforms such as Reddit, Facebook, Tumblr, Twitter and Youtube. During the project, we tracked data to discover how climate change communication takes place across different social media platforms through words, images and videos. I worked with data scientists to help visualise, aggregate and present the data that was collected. The end product of the project was a website for identifying trends and providing data for people to explore.
\end{flushleft}

\section{Ethical issues}

\subsection{Intelectual property}
\begin{flushleft}
Intellectual property is any unique product of the human intellect that has commercial value 
In recent years there has been a massive increase of content generated from all of the social media platfroms. This has also le
\cite{Doe:2009:Online}
\end{flushleft}

\begin{flushleft}

\end{flushleft}

\section{References}

\printbibliography

\end{document}
