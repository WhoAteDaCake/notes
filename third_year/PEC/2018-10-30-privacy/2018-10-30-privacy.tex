\documentclass{article}

\usepackage[%
    left=0.5in,%
    right=0.5in,%
    top=0.5in,%
    bottom=0.5in,%
]{geometry}%
\usepackage{minitoc}
\usepackage{multicol}
\usepackage{graphicx}
\usepackage{fixltx2e}
\usepackage{listings}
\usepackage{color}
\usepackage{hyperref}
    \hypersetup{ colorlinks = true, linkcolor = blue }
\usepackage{blindtext}
\definecolor{lightgray}{gray}{0.9}
\graphicspath{ {./} }

\newcommand{\inlinecode}[2]{\colorbox{lightgray}{\lstinline
[language=#1]$#2$}}
\newcommand{\worddef}[1]{\hyperref[sec:reference]{\textit{#1}}}

\begin{document}

\tableofcontents

\newpage

\section{Harms of privacy}
\begin{itemize}
  \item Taking advantage of privacy (no access) to conduct illegal or immoral activities
  \item Violence in private circles (e.g., families, fraternities, etc.) not perceived from outside
  \item Unhappiness with too great of a burden for a nuclear family to care for all its members
  \item Need for engagement is natural and for those who are outcast (disabled, poor, etc.) “privacy” is a disadvantage
\end{itemize}

\section{Benefits of privacy}
\begin{itemize}
  \item Privacy enables us to establish individuality.
  \item We are responsible for the development of our unique persona, as a separate moral agent.
  \item Privacy is a recognition of a person’s true freedom.
  \item It lets us be ourselves and remove our public persona.
  \item Privacy allows us to shut out the rest of the world and focus on our thoughts, be creative and grow spiritually.
  \item We form different kinds of social relationship and need to have control over who knows what about us.
\end{itemize}

\section{Taxonomy of privacy problems}

\begin{flushleft}
Information collection
\begin{itemize}
  \item Surveillance Monitoring continuously, via audio, visual and computing tech.
  \item Interrogation ”Pressuring individuals to divulge information “
\end{itemize}
Information processing
\begin{itemize}
  \item Aggregation Collecting and linking data
  \item Identification Connecting information to individuals
  \item Insecurity Inadequately safeguarding data
  \item Secondary use Repurposing data disclosed by individuals
  \item Exclusion Failing to notify that data has been collected and allowing to correct records/information
\end{itemize}
Information dissemination
\begin{itemize}
  \item Breach of confidentiality Breaking a contractual duty to keep information private
  \item Disclosure Publishing private and true information in way it damages reputation
  \item Exposure Publicly displaying physical and emotional attributes that are private
  \item Increased accessibility Making records easier to access
  \item Blackmail Threatening to reveal damaging information
  \item Appropriation Using someone’s identity
  \item Distortion Manipulating the way a person is perceived
\end{itemize}
Invasion
\begin{itemize}
  \item Intrusion Disturbing somebody’s peace
  \item Decisional interference Controlling what one is allowed to do.
\end{itemize}
\end{flushleft}

\section{Privacy respecting principles}
\begin{itemize}
  \item Informed consent (opt out vs. opt in)
  \item No invisible information gathering
  \item No secondary use
  \item No covert data mining, matching or profiling
  \item Only collect as needed
  \item Accuracy and security
  \item Policy for responding to data requests
  \item Constitutional protection and laws
  \item Right to be forgotten
  \item Liberty vs. Claim rights
\end{itemize}

\section{OECD}

\begin{flushleft}
Organization for Economic Cooperation and Development. Has issued Recommendations of the Council Concerning Guidelines Governing
the Protection of Privacy and Trans-Border Flows of Personal Data
\end{flushleft}

\subsection{Guidelines}
\begin{itemize}
  \item Notice: data subjects should be given notice when their data is being collected
  \item Purpose: data should only be used for the purpose stated and not for any other purposes
  \item Consent: data should not be disclosed without the data subject’s consent
  \item Security: collected data should be kept secure from any potential abuses
  \item Disclosure: data subjects should be informed as to who is collecting their data
  \item Access: data subjects should be allowed to access their data and make corrections to any inaccurate data
  \item Accountability: data subjects should have a method available to them to hold data collectors accountable for not following the above principles
\end{itemize}

\pagebreak
\section*{Reference section} \label{sec:reference}
\begin{description}
	\item[placeholder] \hfill \\
\end{description}
\end{document}
