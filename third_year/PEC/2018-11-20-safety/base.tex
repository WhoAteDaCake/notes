\documentclass{article}

\usepackage[%
    left=0.5in,%
    right=0.5in,%
    top=0.5in,%
    bottom=0.5in,%
]{geometry}%
\usepackage{minitoc}
\usepackage{multicol}
\usepackage{graphicx}
\usepackage{fixltx2e}
\usepackage{listings}
\usepackage{color}
\usepackage{hyperref}
    \hypersetup{ colorlinks = true, linkcolor = blue }
\usepackage{blindtext}
\definecolor{lightgray}{gray}{0.9}
\graphicspath{ {./} }

\newcommand{\inlinecode}[2]{\colorbox{lightgray}{\lstinline
[language=#1]$#2$}}
\newcommand{\worddef}[1]{\hyperref[sec:reference]{\textit{#1}}}

\begin{document}

\tableofcontents

\newpage

\section{Safety and releaiblity}

\begin{flushleft}
Computing system errors stem from human errors. Errors are typically traceable but moral and legal responsibility is not always easy to establish
\end{flushleft}

\section{Causes of Failure in Computing Systems}

\begin{itemize}
  \item Hardware errors 
  \item Software errors 
  \item Solving the wrong problem 
  \item Misuse of the system 
  \item Human communication failure 
  \item Human malice
\end{itemize}

\section{Dependability}

\begin{flushleft}
\textbf{Dependability} of a computing system is the ability to deliver service that can justifiably be trusted. Dependability attributes are:
\begin{itemize}
  \item Availability: Readiness of correct service 
  \item Reliability: Continuity of correct service 
  \item Safety: Absence of catastrophic consequences on the user(s) and the environment 
  \item Confidentiality: Absence of unauthorized disclosure of information 
  \item Integrity: Absence of improper system state alterations 
  \item Maintainability: Ability to undergo repairs and modifications.
\end{itemize}
\end{flushleft}

\section{Software Engineering}

\begin{flushleft}
Discipline focussed on the production of software and development of tools methodologies and theories supporting software production.
\begin{itemize}
  \item \textbf{Specification}: Defining the functions to be performed by the software 
  \item \textbf{Development}: Producing the software that meets the specifications
  \item \textbf{Validation}: Testing the software
  \item \textbf{Evolution}: Modifying the software to meet the changing needs of the customer
\end{itemize}
\end{flushleft}

\subsection{Specification}

\subsubsection{Process}

\begin{itemize}
  \item Determine the requirements of the system and constraints under which the software must operate   
  \item Assess feasibility of the software development within the budget and schedule requirements
\end{itemize}

\subsubsection{Result}

\begin{itemize}
  \item High level statement of requirements 
  \item Mock-ups of the user interface 
  \item Low-level requirements statement.
\end{itemize}

\subsection{Development}

\begin{flushleft}
First design phase—based on high-level, abstract view of the system
\begin{itemize}
  \item Reveals ambiguities, omissions, and errors in the specifications
  \item Mistakes are corrected; that is less expensive at a higher, abstract level
\end{itemize}
Subsequent design phases—add levels of details
\begin{itemize}
  \item Components of the system become clear
  \item Interface between each components are spelled out 
  \item Algorithms are selected and data structure defined.
\end{itemize}
\end{flushleft}

\subsection{Validation}

\begin{flushleft}
Software testing
\begin{itemize}
  \item Process to assess the correctness, completeness and quality of developed computer software. Validate that software satisfies the specification and meets user needs.
\end{itemize}
Testing can reveal bugs but cannot prove that the program \textbf{will work correctly under all circumstance}.
\begin{itemize}
  \item Formally proving that the software meets specifications may be unfeasible and is typically expensive 
  \item Even if we prove that the program is ‘correct’ and meets specifications that does not mean that specifications are correct.
\end{itemize}
\end{flushleft}

\subsubsection{Principles}

\begin{itemize}
  \item Testing shows presence of defects
  \item Exhaustive testing is impossible: Aim: Optimal amount of testing based on the risk assessment.
  \item Defect clustering: The risk is determined based on experience and critical thinking. In practice, a small number of modules contain most of the defects detected.
  \item Pesticide paradox: If the same tests are repeated they can no longer find new bugs. Tests need to be reviewed, revised, and changed.
  \item Testing is context dependent
  \item Absence of errors is fallacy 
  \item Early testing is critical: Testing for bugs is useless if the system does not meet the user’s requirements. Testing should start as early as possible to ensure that it meets user needs.
\end{itemize}

\subsection{Evolution}

\begin{itemize}
  \item Successful software systems evolve over time to meet the changing needs of the users
  \item Development of new software versions
  \begin{itemize}
    \item New needs are compared with the strengths and weaknesses of the system 
    \item Decision is made about the changes to the system 
    \item Often the same CASE tools are used to support the development of the new version 
    \item Previous test suites may be reused and expanded to test the new version of the system.
    \end{itemize}  
\end{itemize}


\pagebreak
\section*{Reference section} \label{sec:reference}
\begin{description}
	\item[placeholder] \hfill \\
\end{description}
\end{document}
