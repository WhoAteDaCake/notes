\documentclass{article}

\usepackage[%
    left=0.5in,%
    right=0.5in,%
    top=0.5in,%
    bottom=0.5in,%
]{geometry}%
\usepackage{minitoc}
\usepackage{multicol}
\usepackage{graphicx}
\usepackage{fixltx2e}
\usepackage{listings}
\usepackage{color}
\usepackage{hyperref}
    \hypersetup{colorlinks = true, linkcolor = blue}
\usepackage{blindtext}
\definecolor{lightgray}{gray}{0.9}
\graphicspath{ {./} }

\newcommand{\inlinecode}[2]{\colorbox{lightgray}{\lstinline
[language=#1]$#2$}}
\newcommand{\worddef}[1]{\hyperref[sec:reference]{\textit{#1}}}

\begin{document}

\tableofcontents

\newpage

\section{Society, morality and Ethics}
\begin{flushleft}
\textbf{Society:} An association of people organized under a system of rules designed to advance the good of its members over time.\\ \textbf{Morality:} Set of rules of conduct describing what people ought and ought not to do in various situation.\\
\textbf{Ethics:} is the philosophical study of morality and a rational examination into people’s moral beliefs and behaviour. It is focused on the \textbf{voluntary and moral} choices people make when they must make decisions and choose between two or more alternative actions.
\end{flushleft}

\section{Ethics theories}

\subsection{Relativism}
\subsubsection{About}
\begin{flushleft}
Relativism is the theory that there are \textbf{no universal moral} norms of right and wrong. 
\end{flushleft}
\begin{itemize}
  \item Subjective Relativism
  \begin{itemize}
    \item The theory that \textbf{each person} decides ‘right’ and ‘wrong’ for himself or herself.
    \item Popular expression: What is right for you may not be right for me.
  \end{itemize}
  \item Cultural Relativism 
  \begin{itemize}
    \item Is the ethical theory that meaning of ‘right’ and ‘wrong’ rests with a \textbf{society’s moral guidelines} and vary from place to place and from time to time.
    \item For example, in some countries like China, it is acceptable to stare at others in public, or to stand very close to others in public spaces.
  \end{itemize}
\end{itemize}

\subsubsection{Subjective realism}
\begin{flushleft}
  \textbf{Pros:} Well-meaning and intelligent people can have totally opposite opinions about moral issues. Ethical debates are disagreeable and pointless, never reaching an agreement\\
  \textbf{Cons:} No sharp line between doing what you \textbf{think is right and doing what you want}. No moral distinction between actions of different people (since there is no firm reference point) Subjective relativism ≠ tolerance. Tolerance allows individuals in a pluralistic society to live in harmony. What is some people decide to be intolerant. Allows people to make decision based on means other than reason, e.g., rolling a dice.
\end{flushleft}

\subsubsection{Cultural Relativism}
\begin{flushleft}
  \textbf{Pros:} Different social context demand different moral guidelines. It is arrogant for one society to judge another.\\
  \textbf{Cons:} No clear mechanisms for establishing what the moral guidelines are for a particular culture. Individuals may do it by induction–incomplete and unreliable. There may not be clearly established norms due to disagreements among groups No framework for reconciliation between cultures in conflict Existence of many cultural practices does not imply that any is acceptable (many/any fallacy)
\end{flushleft}


\subsection{Divine command theory}
\begin{flushleft}
Good and bad actions are those aligned with the will of God. Use Holy books as moral decision making guides. 
\begin{itemize}
  \item Pros: God is all-good and allknowing God is the ultimate authority We owe obedience to our Creator.
  \item Cons: Based on obedience. Not reason. There are many Holy books and many disagree. Multicultural societies are unlikely to adopt a religion based morality Holy books do not address all the moral problems.
\end{itemize}
\end{flushleft}

\pagebreak

\subsection{Ethical Egoism}
\begin{flushleft}
Each person should focus \textbf{exclusively} on his or her self interest. Morally right action for a person is the action that provides the maximum long-term benefit for the person.
\end{flushleft}
\begin{itemize}
  \item Pros: It is a rational theory. We are naturally inclined to do what is best for ourselves. Better to let people take care of themselves—they know better what they want and need. Communities benefit from individual who put their well-being first—often make environment better for others. Other moral principles are based on self interest (e.g., ‘do not break a promise’).
  \item Cons: Easy moral philosophy; may not e the best. We do know a lot about what is good for someone else A self-interested focus can lead to immoral behaviour. Other moral principles are superior to the principle of self interest People who take the good of others into account live happier lives
\end{itemize}

\subsection{Social Contract Theory}
\begin{flushleft}
Morality consists in rules, governing how people are to treat one another, that rational people will agree to accept for their mutual benefit, on the condition that others follow those rules as well. Everybody living in a civilized society has implicitly agreed to:
\begin{itemize}
  \item The establishment moral rules to govern relations among citizens 
  \item A government capable of enforcing the social rules
\end{itemize}
\end{flushleft}

\subsubsection{Social Contract Concepts and Ideas}
\begin{flushleft}
Can the society is finding a form of association that \textbf{guarantees everybody their safety} and property, \textbf{yet enables each person to remain free}? There is a close correspondence between RIGHTS and DUTIES. E.g., the right to life means everyone’s duty to protect life.
\begin{itemize}
  \item Negative right means that one cannot interfere in exercising right by another person. E.g., the right of free expression.
  \item Positive right obligates others to do something on one’s behalf, e.g., free education. The Society must reserve resources to provide/support that right.
\end{itemize}
\end{flushleft}

\subsubsection{Inequality}
Each person may claim rights and liberties so long as these claims are consistent with everyone else having the same rights and liberties. Any social and economic inequality must satisfy two conditions:
\begin{itemize}
  \item Inequality is associated with positions in society that everyone has a fair and equal opportunity to assume
  \item Inequalities must be to the greatest benefit of the leastadvantaged members of society (the difference principle).
\end{itemize}

\section{Obligations vs Consequences}
\begin{flushleft}
\textbf{Deontology (Kantianism)} – actions guided and rationally explained by rights, duty, and obligations (regardless of whether consequences are good or bad)\\
\textbf{Utilitarianism} – actions are guided by the type of consequences (and not by rights, duty and obligations), and maximising happiness or social utility
\end{flushleft}

\subsection{Consequentialism}
\begin{flushleft}
Consequences of an action makes the action good or bad (not the motivation behind the action). In utilitarianism: right decision is the one that causes the most happiness.
\end{flushleft}

\subsection{Deontological ethics theories}
\begin{flushleft}
Focus on will, rights, duties, obligations, rules Some rules must be followed, even if they result in a bad outcome.
\end{flushleft}

\subsubsection{Act deontology}
\begin{flushleft}
Call on individuals to “Do the right thing!” Humans possess moral intuition by which they perceive their obligations. Intuition is exhibited in specific situations and through norms established through one’s life/existence.
\end{flushleft}

\subsection{Rule deontology}
\begin{flushleft}
Based on rules (e.g., "Do not lie"), rather than on particular judgments. Humans possess a capacity to reason and learn their obligations towards one another. Kantianism is an example of the Rule Deontological Theory
\end{flushleft}

\subsection{Kantianism}
\begin{flushleft}
The only good is a ‘good will’. Categorical imperative is a rule that must be followed.
\begin{itemize}
  \item Universal law of nature: Must act only according to maxims (principles) that could be adopted as universal laws.
  \item End in itself: Treat humans, both yourself and others, as ends in themselves and never as a means to an end.
\end{itemize}
\end{flushleft}

\pagebreak
\section*{Reference section} \label{sec:reference}
\begin{description}
	\item[placeholder] \hfill \\
\end{description}
\end{document}
