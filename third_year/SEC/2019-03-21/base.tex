\documentclass{article}

\usepackage[%
    left=0.5in,%
    right=0.5in,%
    top=0.5in,%
    bottom=0.5in,%
]{geometry}%
\usepackage{minitoc}
\usepackage{multicol}
\usepackage{graphicx}
\usepackage{fixltx2e}
\usepackage{listings}
\usepackage{color}
\usepackage{hyperref}
    \hypersetup{ colorlinks = true, linkcolor = blue }
\usepackage{blindtext}
\definecolor{lightgray}{gray}{0.9}
\graphicspath{ {./} }

\newcommand{\inlinecode}[2]{\colorbox{lightgray}{\lstinline
[language=#1]$#2$}}
\newcommand{\worddef}[1]{\hyperref[sec:reference]{\textit{#1}}}

\begin{document}

\tableofcontents

\newpage

\section{Cookies}
\begin{itemize}
	\item HTTP is a stateless protocol, but most of what we do online is stateful 
	\item Cookies are small text files used to provide persistence
	\item Servers can provide cookies during HTTP responses, using Set-Cookie 
	\item Browsers will return any cookies for a given domain in GET and POST requests
\end{itemize}

\subsection{Types of Cookie}
\begin{itemize}
	\item Session – \textbf{Deleted when the browser exits}, contain no expiration date 
	\item Persistent – Expire at a given time 
	\item Secure – Can only be used over HTTPS 
	\item HTTPOnly – Inaccessible from JS
\end{itemize}

\subsection{Third Party Cookies}
\begin{itemize}
	\item Cookies are associated with the domains that produced them. Amazon.com cookies don’t go to google.co.uk 
	\item Some websites include request to other domains, such as 3rd party advertisers. These serve cookies – a lot
\end{itemize}

\subsection{Cookie Vulnerabilities}
\begin{itemize}
	\item How a website uses a cookie is up to the server 
	\item Many create an SID to authenticate users, for example to “keep me logged on” 
	\item Obtaining this cookie – Cookie Stealing – lets you hijack their session 
	\begin{itemize}
		\item HTTP Cookies can be stolen simply by monitoring 
		\item HTTPS will require Cross-site scripting attacks or DNS poisoning
	\end{itemize}
\end{itemize}

\section{SSL/TLS}
\begin{itemize}
	\item There are dangers associated with sending plain text cookies, passwords etc. 
	\item SSL, and the newer TLS provide authenticated and encrypted sessions 
	\item Secure Socket Layer (SSL) came first, then after v3.0 it became Transport Layer Security (TLS), currently v1.2 / v1.3
\end{itemize}

\subsection{TLS}
Transport Layer Security has two layers:
\begin{itemize}
	\item The Record Layer: using established symmetric keys and other session info, will encrypt application packets, very like IPSec 
	\item The Handshake Layer: used to establish session keys, as well as authenticate either party – usually the server using a public-key certificate
\end{itemize}

\subsection{TLS handshake}
\begin{flushleft}
Look at \href{https://moodle.nottingham.ac.uk/pluginfile.php/5124101/mod_resource/content/0/L15%20-%20Internet%20Security%202pp.pdf}{notes}
\end{flushleft}

\subsection{TLS Vulnerabilities}
\begin{itemize}
	\item The man-in-the-middle vulnerabilities are usually countered using public-key authentication 
	\item The majority of TLS problems are implementation: Heartbleed 
	\item Protocol downgrade attacks are still a concern – many servers still allow weak cipher suites. FREAK and Logjam force the use of 512-bit keys
\end{itemize}

\section{Cross-site Scripting (XSS)}
\begin{itemize}
	\item A type of injection attack, similar in many ways to an SQL Injection 
	\item HTML is read by a browser, and is a combination of content (text) and structure (html tags) 
	\item If we can inject html structures into the content of a website, the browser will simply execute these – e.g. <script> tags
\end{itemize}

\subsection{Reflected XSS}
\begin{flushleft}
A malicious URL that inserts an exploit directly into the page returned by a server
\end{flushleft}

\subsection{Persistent XSS}
\begin{itemize}
	\item Even worse, no need to trick people into clicking links 
	\item Any website that doesn’t properly sanitise html tags from user input is vulnerable 
	\item Blog posts with comment sections are obvious targets, but there are many more 
	\item Forums, web comments, shopping reviews, etc
\end{itemize}

\subsection{Preventing XSS}
\begin{itemize}
	\item Websites must aggressively escape HTML characters from any user input / output 
	\item You also need to find all of the bizarre obfuscated versions of XSS 
	\item When you consider all of the things people input on interactive websites, this can be a real problem
\end{itemize}

\subsection{Cross-site request forgery}
\begin{itemize}
	\item When a user puts in an HTTP request, they will also send any relevant session cookies. E.g. an SID from having logged in 
	\item If the user has already authenticated, a malicious URL can then perform some action on their account
\end{itemize}

\subsection{Cross-site Request Forgery (XSRF)}
\begin{itemize}
	\item When a user puts in an HTTP request, they will also send any relevant session cookies. E.g. an SID from having logged in 
	\item If the user has already authenticated, a malicious URL can then perform some action on their account
\end{itemize}

\subsection{XSRF in POST}
\begin{itemize}
	\item Most websites use POST, this is little defence 
	\item The phishing email just points to a convincing website with a malicious form on it
\end{itemize}

\subsection{Preventing XSRF}
\begin{itemize}
	\item XSS vulnerabilities make XSRF a lot easier! Fix these! 
	\item Use synchroniser tokens 
	\item Each website form has a one-time token that the server validates when the form is submitted
\end{itemize}

\pagebreak
\section*{Reference section} \label{sec:reference}
\begin{description}
	\item[placeholder] \hfill \\
\end{description}
\end{document}
