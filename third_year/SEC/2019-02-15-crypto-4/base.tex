\documentclass{article}

\usepackage[%
    left=0.5in,%
    right=0.5in,%
    top=0.5in,%
    bottom=0.5in,%
]{geometry}%
\usepackage{amsmath}
\usepackage{minitoc}
\usepackage{multicol}
\usepackage{graphicx}
\usepackage{fixltx2e}
\usepackage{listings}
\usepackage{color}
\usepackage{hyperref}
    \hypersetup{ colorlinks = true, linkcolor = blue }
\usepackage{blindtext}
\definecolor{lightgray}{gray}{0.9}
\graphicspath{ {./} }

\newcommand{\inlinecode}[2]{\colorbox{lightgray}{\lstinline
[language=#1]$#2$}}
\newcommand{\worddef}[1]{\hyperref[sec:reference]{\textit{#1}}}

\begin{document}

\tableofcontents

\newpage

\section{Integer Factorisation}
\begin{flushleft}
Any integer can be expressed as the multiplication of a list of prime numbers. The longer the value, the harder (and slower) this gets. \textit{Semi-primes} (product of two \textbf{primes}) are the hardest numbers to factor
\end{flushleft}

\section{RSA}
\begin{itemize}
  \item RSA is the most common method for general public key cryptography 
  \item It provides both encryption and/or authentication 
  \item RSA provides us with two keys:
  \begin{itemize}
    \item Public (e, n). e is usually a small number, d is a much larger number n is a \textbf{very large semi-prime number}
    \item Private d
  \end{itemize}
  \item The values e and d are mathematically linked such that:
  \begin{gather}
   M^e = C \kern-0.3cm \pmod{n} \\
   C^d = M \kern-0.3cm \pmod{n}
  \end{gather}
  \item They are inverses of one another, when used as exponents $ \kern-0.2cm \mod{n}$
\end{itemize}

\subsection{Euler Totient Function}
\begin{itemize}
  \item Integers a and b are \textit{relatively prime} if they \textbf{do not share a divisor} (except 1) 
  \item The Euler totient $\phi$ is the integers from 1 to n-1 that are relatively prime with n
  \item The \textbf{totient value} of a prime p is simply p-1 
  \item For two primes multiplied together it’s (p-1)(q-1)
\end{itemize}

\subsection{RSA key generator}
\begin{itemize}
  \item Choose two large primes, p and q, then calculate n = pq
  \item Select a value e that is relatively prime with the totient of n
  \begin{gather}
    p = 17, q = 11 \\
    n = p * q = 187 \\
    \phi(n) = (p - 1)*(q - 1) = 160 \\
    e =  one \; of \; 3, 6, 7, 11 = 7
  \end{gather}
  \item p, q, $\phi(n)$ - Private
  \item e, n - Public
  \item Calculate a multiplicative inverse to e: d, where
    \[ (e * d) \kern-0.3cm \mod{\phi(n)} = 1 \]
  \item This is easily achieved if we know $\phi(n)$ , but not otherwise
  \item Now we have a public key e,n and a private key d
\end{itemize}

\subsection{Why is RSA secure}
\begin{itemize}
  \item We need to know $\phi(n)$ to find d
  \item Finding this is extremely hard, for example we could factor n into p and q
\end{itemize}

\subsection{Using RSA}
\begin{itemize}
  \item The keys (e, n) and (d) are reversible – either can be used for encryption, and the other used for decryption
  \item Everyone knows the public key, only the owner knows the private key 
  \item This leads us to two very useful use cases for RSA:
  \begin{itemize}
    \item Encryption only the owner can read  
    \item Signing that must have been performed by the owner
  \end{itemize}
\end{itemize}

\section{Digital Signatures}
\begin{itemize}
  \item Authentication codes provide integrity, \textbf{but don’t guarantee the sender }
  \item Public-key encryption allows us to \textbf{sign documents}
  \item We can use a trusted third party in order to \textbf{verify the ownership of a public key} 
  \item Bob then knows he has Alice’s genuine key, not an imposter 
  \item Can also be ‘self signed’ 
  \item An important part of Transport Layer security (TLS)
\end{itemize}

\subsection{X.509 v3}
\begin{itemize}
  \item Organised by Public Key Infrastructure (PKI) 
  \item The standard for digital certificates holds information on the type, subject and issuer 
  \item The issuer will usually be a \textbf{trusted third party}, like Verisign or Globalsign
\end{itemize}

\subsection{Certificate issuance}
\begin{itemize}
  \item Server (server.com) has a certificate containing their public key, which they want people to trust
  \item They go to a certificate authority (CA), who after doing ID checks, sign the certificate with their private key
  \item The server can supply digital signatures using the public key, backed by the certificate when requested, e.g. during a TLS handshake
  \item To verify the trust in the Server.com certificate, we need to examine the signing certificate
  \item In many cases, the chain involves multiple certificates 
  \item Chains always end in a root certificate, \textbf{located on your machine}
\end{itemize}

\pagebreak
\section*{Reference section} \label{sec:reference}
\begin{description}
	\item[placeholder] \hfill \\
\end{description}
\end{document}
