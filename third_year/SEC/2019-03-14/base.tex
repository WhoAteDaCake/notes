\documentclass{article}

\usepackage[%
    left=0.5in,%
    right=0.5in,%
    top=0.5in,%
    bottom=0.5in,%
]{geometry}%
\usepackage{minitoc}
\usepackage{multicol}
\usepackage{graphicx}
\usepackage{fixltx2e}
\usepackage{listings}
\usepackage{color}
\usepackage{hyperref}
    \hypersetup{ colorlinks = true, linkcolor = blue }
\usepackage{blindtext}
\definecolor{lightgray}{gray}{0.9}
\graphicspath{ {./} }

\newcommand{\inlinecode}[2]{\colorbox{lightgray}{\lstinline
[language=#1]$#2$}}
\newcommand{\worddef}[1]{\hyperref[sec:reference]{\textit{#1}}}

\begin{document}

\tableofcontents

\newpage

\section{IP security}

IP is connectionless and stateless
\begin{itemize}
	\item Best effort service 
	\item No delivery guarantee 
	\item No order guarantee 
	\item IPv4 No guaranteed security support 
	\item IPv6 security support is guaranteed - IPSec.
\end{itemize}

\subsection{IPSec}
\begin{itemize}
	\item Optional in IPv4, mandatory support in IPv6 
	\item Two major security mechanisms 
	\item IP Authentication Header (AH). (Not really used, because it's not to useful)
	\item IP Encapsulation Security Payload (EPS). We both encrypt and authenticate data
	\item Does not contain any mechanisms to prevent traffic analysis.
\end{itemize}

\subsection{Encapsulation Security Payload (ESP)}
\begin{flushleft}
Includes an additional header within the IP packet that describes what encryption and authentication is in use
\end{flushleft}


\pagebreak
\section*{Reference section} \label{sec:reference}
\begin{description}
	\item[placeholder] \hfill \\
\end{description}
\end{document}
