\documentclass{article}

\usepackage[%
    left=0.5in,%
    right=0.5in,%
    top=0.5in,%
    bottom=0.5in,%
]{geometry}%
\usepackage{minitoc}
\usepackage{multicol}
\usepackage{graphicx}
\usepackage{fixltx2e}
\usepackage{listings}
\usepackage{color}
\usepackage{hyperref}
    \hypersetup{ colorlinks = true, linkcolor = blue }
\usepackage{blindtext}
\definecolor{lightgray}{gray}{0.9}
\graphicspath{ {./} }

\newcommand{\inlinecode}[2]{\colorbox{lightgray}{\lstinline
[language=#1]$#2$}}
\newcommand{\worddef}[1]{\hyperref[sec:reference]{\textit{#1}}}

\begin{document}

\tableofcontents

\newpage

\section{Usernames and Passwords}
\begin{itemize}
  \item Identification – Who you are 
  \item Authentication – Verify that identity 
  \item Authentication \textbf{should expire}. “Remember my credentials” turns this into something you have 
  \item Time of check to time of use – TOCTTOU 
  \begin{itemize}
    \item Repeated authentication 
    \item At the start and during a session
  \end{itemize}
\end{itemize}

\subsection{Problems With Passwords}
\begin{itemize}
  \item People forget them 
  \item They can be guessed
  \item Spoofing and Phishing (pretend to be a different website)
  \item Compromised password files 
  \item Keylogging 
  \item Many of these are made many times worse by weak passwords
\end{itemize}

\subsection{Problems with password policies}
\begin{itemize}
  \item People attempt to make their life easier by re-using passwords 
  \item When they’re forced to change to unique passwords, they’ll simply increment a counter
\end{itemize}

\subsection{Password shadow files}
\begin{itemize}
  \item Operating systems have taken steps to stop people reading hashes for offline attacks 
  \item These files are now \textbf{read protected }
  \item Administrators or people booting another OS will often find a way in
\end{itemize}

\section{Cracking passwords}
\begin{itemize}
  \item Password cracking falls into two basic types: 
    \begin{itemize}
      \item Offline: You have a copy of the password hash locally 
      \item Online: You do not have the hash, and are instead attempting to gain access to an actual login terminal 
    \end{itemize}
  \item Online is usually attempted with phishing
  \item Offline password cracking quite simply a case of trying possible passwords, and seeing if we have a hash collision with a password list. Might be a \textbf{brute force approach}
\end{itemize}

\subsection{Dictionary Attacks}
\begin{itemize}
  \item Most password cracking is now achieved using dictionary attacks rather than brute force 
  \item Using a dictionary of common words and passwords 
  \item Apply small variations to this list, trying them all 
  \item Combine words from two different lists
\end{itemize}

\section{Password Salting}
\begin{itemize}
  \item We can improve security by prepending a random “salt” to a password before hashing 
  \item The salt is stored unencrypted with the hash
  \item If we use a different random salt for each user, we get the following security benefits:
  \begin{itemize}
    \item Cracking multiple passwords is slower – a hit is for a single user, not all users with that password
    \item Prevents rainbow table attacks – we can’t pre-compute that many password combinations 
  \end{itemize}
  \item Salting has no effect on the speed of cracking a single password – so make your passwords good!
\end{itemize}

\section{Hashing Speed}
\begin{itemize}
  \item When password cracking, the most important factor is hashing speed 
  \item Newer algorithms take longer: partly because they’re more complex, but some have been specifically designed to take a while 
  \item Iterate to increase complexity - PBKDF2 
  \item bcrypt can’t be used on easily GPUs
\end{itemize}

\section{Pretexting}
\begin{itemize}
  \item Obtaining private details by offering some “pretext” as a reason for needing them 
  \item We continue to rely on email addresses, DOB and Mother’s maiden names as our “last line of defense” for security.
  \item What are alternatives?
\end{itemize}

\section{Multi-factor authentication}  
\begin{itemize}
  \item Combines something you \textbf{know} with something you \textbf{have} 
  \item Common examples: 
  \item Text codes to mobiles 
  \item One time passwords, Google Authenticator, Microsoft Authenticator etc. 
  \item USB devices e.g. Yubico 
  \item New devices and TOCTTOU are a common uses for two-factor authentication
\end{itemize}

\section{Biometrics}
\begin{itemize}
  \item Measurements of the human body, something you are 
  \item Various forms, fingerprint recognition, iris / retina recognition, voice, gait, typing rhythm 
  \item A password you always have with you, but you can’t change 
  \item Usually a trade off between false positives and negatives
\end{itemize}

\pagebreak
\section*{Reference section} \label{sec:reference}
\begin{description}
	\item[placeholder] \hfill \\
\end{description}
\end{document}
