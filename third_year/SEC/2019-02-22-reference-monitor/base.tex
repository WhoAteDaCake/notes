\documentclass{article}

\usepackage[%
    left=0.5in,%
    right=0.5in,%
    top=0.5in,%
    bottom=0.5in,%
]{geometry}%
\usepackage{minitoc}
\usepackage{multicol}
\usepackage{graphicx}
\usepackage{fixltx2e}
\usepackage{listings}
\usepackage{color}
\usepackage{hyperref}
    \hypersetup{ colorlinks = true, linkcolor = blue }
\usepackage{blindtext}
\definecolor{lightgray}{gray}{0.9}
\graphicspath{ {./} }

\newcommand{\inlinecode}[2]{\colorbox{lightgray}{\lstinline
[language=#1]$#2$}}
\newcommand{\worddef}[1]{\hyperref[sec:reference]{\textit{#1}}}

\begin{document}

\tableofcontents

\newpage

\section{Reference Monitor}
\begin{flushleft}
An access control concept that refers to an abstract machine that
\textbf{mediates all access} to objects by subjects
\end{flushleft}
\begin{itemize}
  \item Must be tamper proof / resistant 
  \item Must \textbf{always be invoked} when access to an object is required 
  \item Must be small enough to be verifiable / subject to analysis to ensure correctness
\end{itemize}

\section{Placement}

\begin{flushleft}
Can be placed anywhere within the system, a variety of locations relative to the program being run
\begin{itemize}
  \item Hardware – Dedicated registers for defining privileges 
  \item Operating system kernel – E.g. Virtual Machine Hypervisor 
  \item Operating system – Windows security reference monitor 
  \item Services Layer – JVM, .NET 
  \item Application Layer – Firewalls
\end{itemize}
\end{flushleft}

\subsection{Lower Is Better}
\begin{flushleft}
Using a reference monitor or other security features at a \textbf{lower level} means:
\begin{itemize}
  \item We can assure a higher degree of security 
  \item Usually simple structures to implement 
  \item Reduced performance overheads 
  \item Fewer layer below attack possibilities
\end{itemize}
However: \textbf{access control decisions} are far removed from applications
\end{flushleft}

\section{OS Integrity}
\begin{flushleft}
The operating system: arbitrates access requests, is itself an resource that must be accessed. This is a \textbf{conflict}, we want to \textbf{use} the OS but \textbf{not mess with it}.
\end{flushleft}

\subsection{Modes of operation}
\begin{flushleft}
Defines which actions are permitted in which mode, e.g. system calls, machine instructions, I/O. Distinguish between computations done on behalf of: \textbf{the OS and the user}. A \textit{status flag} within the CPU allows the OS to operate in different modes
\end{flushleft}

\subsection{Controlled Invocation}
\begin{flushleft}
Many functions are held at kernel level, but are quite reasonably called from within user level code 
\begin{itemize}
  \item Network and File IO 
  \item Memory allocation 
  \item Halting the CPU (at shutdown only!) 
\end{itemize}
We need a mechanism to transfer between kernel mode (ring 0 - root) and user mode (ring 3 - user)
\end{flushleft}

\subsubsection{Controlled Invocation: Interrupts}
\begin{flushleft}
Also known as exceptions / traps. In many ways is the \textbf{hardware equivalent} to a \textbf{software exception} – not always bad. Handled by an \textit{interrupt handler} which resolves the issue and returns to the original code. Given an interrupt, the CPU will \textbf{switch execution} to the location given in an \textit{interrupt descriptor table}
\end{flushleft}

\section{Descriptors and Selectors}
\begin{itemize}
  \item Descriptors hold information on crucial system objects like \textbf{kernel structure locations} 
  \item Descriptors are held in \textit{descriptor tables} 
  \item Contain a Descriptor Privilege Level (DPL) 
  \item Descriptors are indexed by selectors 
  \item Loaded when required, e.g. on jump calls 
  \item The CPU protects the kernel by \textbf{checking} the Current Privilege Level (CPL) when a Selector is loaded
\end{itemize}

\subsection{Interrupt-gates}
\begin{itemize}
  \item The code segment (CS) register in x86 CPUs has 2 bits reserved for the Current Privilege Level (CPL) 
  \item Descriptors that have a privilege level \textbf{higher than where they point are called} gates 
  \item Since these descriptors are \textbf{created by the kernel}, they offer a \textbf{secure means of entry into ring 0}
\end{itemize}

\subsection{Patching the kernel}
\begin{flushleft}
If you can run custom PL 0 code (compromised driver?), you can insert your own handler – \textit{Rootkit}
\end{flushleft}

\section{Processes and Threads}
\begin{itemize}
  \item A process is a program being executed. Important unit of control: 
  \begin{itemize}
    \item Exists in its own address space 
    \item Communicates with other processes via the OS 
    \item Separation for security 
  \end{itemize}
  \item A Thread is a strand of execution within a process. Shares a common address space
\end{itemize}

\section{Memory Protection}
\begin{flushleft}
Segmentation – divides data into logical units
\begin{itemize}
  \item Good for security 
  \item Challenging memory management 
  \item Not used much in modern OSs
\end{itemize}
Paging – divides memory into pages of equal size
\begin{itemize}
  \item Efficient memory management 
  \item Worse for access control 
  \item Extremely common in modern OSs
\end{itemize}
\end{flushleft}

\section{Side channel exploits}
\begin{flushleft}
In most operating systems, the entire kernel is stored in the \textbf{upper address space}. Pages in this area are flagged as supervisor, and cannot be accessed in ring 0. In Intel (and some other) CPUs, it’s common to \textbf{speculatively evaluate} code prior to reaching it. If the branch isn't taken, changes are rolled back, however cache isn't.
\end{flushleft}

\subsection{Meltdown}
\begin{flushleft}
Meltdown breaks the mechanism that keeps applications from accessing arbitrary system memory. Consequently, applications can access system memor
\end{flushleft}

\subsection{Spectre}
\begin{flushleft}  
Spectre tricks other applications into accessing arbitrary locations in their memory
\end{flushleft}

\pagebreak
\section*{Reference section} \label{sec:reference}
\begin{description}
	\item[placeholder] \hfill \\
\end{description}
\end{document}
