\documentclass{article}

\usepackage[%
    left=0.5in,%
    right=0.5in,%
    top=0.5in,%
    bottom=0.5in,%
]{geometry}%
\usepackage{minitoc}
\usepackage{multicol}
\usepackage{graphicx}
\usepackage{fixltx2e}
\usepackage{listings}
\usepackage{color}
\usepackage{hyperref}
    \hypersetup{ colorlinks = true, linkcolor = blue }
\usepackage{blindtext}
\definecolor{lightgray}{gray}{0.9}
\graphicspath{ {./} }

\newcommand{\inlinecode}[2]{\colorbox{lightgray}{\lstinline
[language=#1]$#2$}}
\newcommand{\worddef}[1]{\hyperref[sec:reference]{\textit{#1}}}

\begin{document}

\tableofcontents

\newpage

\section{Vectors}
\begin{itemize}
  \item Vectors are the mechanism through which malware infects a machine 
  \item Usually the vector will be a software vulnerability 
  \item Or, someone clicked something they shouldn’t have!
\end{itemize}

\section{Payloads}
\begin{flushleft}
Payloads are the actual malware deposited on the machine, or the harmful results
\end{flushleft}

\section{Viruses}
\begin{itemize}
  \item A virus is a piece of self-replicating code 
  \item Propagates by attaching itself to a disk, file or document 
  \item When the file is run, the virus runs, and attempts to proliferate 
  \item Installs without the user’s knowledge or consent
\end{itemize}

\section{Worms}
\begin{itemize}
  \item Viruses traditionally require a human to spread 
  \item Worms are self-replicating and standalone programs 
  \item Do not require human intervention 
  \item Scanning worms or email worms 
  \item \textbf{Exploit known software vulnerabilities in order to spread}
\end{itemize}

\section{Trojans}
\begin{itemize}
  \item A malicious program pretending to be a legitimate application 
  \item Often obtained in email attachments or at malicious websites 
  \item Don’t replicate themselves – user error 
  \item Ransomware is the most common form of Trojan now
\end{itemize}

\subsection{Ransomware}
\begin{itemize}
  \item Will usually encrypt or block access to files and demand a ransom 
  \item It is a clever solution, because if an AV system removes it, it is often too late 
  \item Usually distributed on malicious websites, or to already infected machines 
  \item The file decryption keys are protected by encrypting using the public key of a C\&C server
\end{itemize}

\subsection{Ransomware Variants}
\begin{flushleft}
Most of the challenge in successfully using ransomware is \textbf{tricking a user into running it}, and bypassing AV and browser protections
\begin{itemize}
  \item Fake emails 
  \item Malicious web pages 
  \item Obfuscated Javascript attachments 
  \item Deployed using exploit kits
\end{itemize}
\end{flushleft}

\section{Anti-virus}
\begin{flushleft}
Signature-based Detection:
\begin{itemize}
  \item Store some small code signature for each virus 
  \item Scan files either in bulk or at runtime, compare with the signatures on file 
  \item Generic signatures
\end{itemize}
Heuristics:
\begin{itemize}
  \item Determine what actions and rules a virus program will normally adopt 
  \item Start the program in a VM and see what it does 
\end{itemize}
\end{flushleft}

\pagebreak
\section*{Reference section} \label{sec:reference}
\begin{description}
	\item[placeholder] \hfill \\
\end{description}
\end{document}
