\documentclass{article}

\usepackage[%
    left=0.5in,%
    right=0.5in,%
    top=0.5in,%
    bottom=0.5in,%
]{geometry}%
\usepackage{minitoc}
\usepackage{multicol}
\usepackage{graphicx}
\usepackage{fixltx2e}
\usepackage{listings}
\usepackage{color}
\usepackage{hyperref}
    \hypersetup{ colorlinks = true, linkcolor = blue }
\usepackage{blindtext}
\definecolor{lightgray}{gray}{0.9}
\graphicspath{ {./} }

\newcommand{\inlinecode}[2]{\colorbox{lightgray}{\lstinline
[language=#1]$#2$}}
\newcommand{\worddef}[1]{\hyperref[sec:reference]{\textit{#1}}}

\begin{document}

\tableofcontents

\newpage

\section{Windows access control}
\begin{itemize}
  \item Windows predominantly uses Access Control Lists. Extends the usual read, write and execute with: 
  \begin{itemize}
    \item Take ownership 
    \item Change permissions 
    \item Delete 
  \end{itemize}
  \item 32-bit access masks (cf. Unix 9-bit). A higher degree of control, with the associated complexity increase!
\end{itemize}

\subsection{Access Control Matrix}
\begin{flushleft}
Access rights are defined \textbf{individually} for each combination of \textbf{subject and object}. Quite an abstract concept, but would allow for very fine grained control. Not practical, think of the \textbf{memory required} in scaling it up!
\end{flushleft}

\subsection{Access Control List}
\begin{flushleft}
Stored with an object itself, corresponding to a column of an ACM. Only need data of relevent subjects. E.g. \textbf{budget.xlsx} Alice: r,w,e Bob: r Claire: r
\end{flushleft}

\subsection{Access Control}
\begin{itemize}
  \item Access control in windows treats more than just files, also:
  \begin{itemize}
    \item Registry keys 
    \item Active directory objects 
    \item Groups
  \end{itemize}
  \item Inheritance is implemented: File can inherit ACLs from parent directories. This allows to set defaults nicely
\end{itemize}

\section{Principals}
\begin{flushleft}
Principals more broadly defined as well: 
\begin{itemize}
\item Local users 
\item Domain users 
\item Groups 
\item Machines (no longer runs on behalf of the user)
\end{itemize}
 Each principal has a human-readable name and security ID (SID)
\end{flushleft}

\subsection{Local / Domain Principals}
\begin{itemize}
\item LSA creates local principals. principal = MACHINE\textbackslash principal 
\item Domain principals administered on DC by domain admins. principal@domain = DOMAIN\textbackslash principal.
\item net user /domain, net group /domain, net localgroup /domain
\end{itemize}

\subsection{Groups}
\begin{itemize}
  \item Groups are collections of SIDs (object-orientated) 
  \item Group can itself be an SID 
  \item Groups can thus be nested 
  \item Groups are not nest-able on local machines 
  \item Managed by a \textbf{domain controller} within Active Directory
\end{itemize}

\section{Objects}
\begin{itemize}
  \item Objects are passive entities in access operations. In Windows: 
  \begin{itemize}
    \item Executive objects (processes, threads, etc.) 
    \item Private objects (files, directories) 
  \end{itemize}
  \item Securable objects have a security descriptor 
  \begin{itemize}
    \item Owner SID
    \item Primary group
    \item DACL - Anyone that has different access rights to the object
    \item SACL - for auding purposes.
  \end{itemize}
  \item Built-in securable objects managed by the OS 
  \item Private objects managed by application software
\end{itemize}

\section{Access Tokens}
\begin{itemize}
  \item Security credentials for a login session stored in access token 
  \item Identifies the user, the user’s groups, and the user’s privileges
  \item Structure:
  \begin{itemize}
    \item User SID
    \item Groups and Alias SID
    \item Privileges
    \item Defaults for New Objects
    \item Miscellaneous
  \end{itemize}
  \item Token is generated during login, meaning if access is revoked, user will still be able to utilise it untill logout.
\end{itemize}

\section{Subjects}
\begin{itemize}
  \item Windows subjects: Processes and threads 
  \item New processes get a copy of the parent access token, possibly modified 
  \item Individual access tokens are immutable, and can live beyond policy changes (TOCTTOU issue)
\end{itemize}

\section{User Account Control}
\begin{itemize}
  \item After Vista, administrator users do not use an administrative access token by default 
  \item Users \textbf{have two tokens}, one heavily restricted and used by default 
  \item A prompt allows a user to spawn a process \textbf{with the other token}, or switch a process’ token
\end{itemize}

\section{Domains}  
\begin{itemize}
  \item Single sign-on for network resources 
  \item Centralised security administration 
  \item Domain Controller (DC): handles user accounts and access control, is a trusted 3rd party for authentication 
  \item Multiple DCs allow for decentralisation by design
\end{itemize}

\subsection{Interactive Logon}
\begin{itemize}
  \item The windows interactive logon allows a user to authenticate 
  \item Windows logon begins with the Secure Attention Sequence: Ctrl + Alt + Delete. Can prevent spoofing - is tied directly to winlogon
\end{itemize}

\subsubsection{Logon process}
\begin{itemize}
  \item Winlogon – the process responsible for authenticating users 
  \item Graphical Identification and Authentication (GINA) 
  \item The Local Security Authority (LSA) 
  \item An authentication package (NTLM and Kerberos) 
  \item Security Account Manager (SAM) 
  \item Since Vista, additional Credential Providers are allowed (2Factor auth, usb auth, etc.)
\end{itemize}

\subsubsection{Domain logon}
\begin{itemize}
  \item Replaces NTLM with Kerberos 
  \item Replaces SAM with an Active Directory Domain Controller 
  \item Checks of a user are now performed on \textbf{the remote LSA}
\end{itemize}

\subsection{Kerberos}
\begin{itemize}
  \item The default authentication for network logon in Windows 
  \item Uses symmetric encryption
  \item Requires a trusted third party
\end{itemize}

\subsubsection{Process}
\begin{itemize}
  \item Contains 2 core services: Authentication server and Ticket granting server
  \item Everyone will have a long term key
  \item When machines want to talk between each other, they obtain a key from a key granting server
  \item look lesson: \href{https://echo360.org.uk/lesson/db83e544-8e6f-4cbe-93d7-c7b6fc92d131/classroom#sortDirection=desc}{here}
\end{itemize}

\subsubsection{Important features}
\begin{itemize}
  \item Including nonces / timestamps prevents replay attacks. But, clocks must be synchronised between principals 
  \item Windows Kerberos buries domain group IDs inside tickets, for access checks 
  \item The ticket granting ticket usually exists until log-off, or rotates daily 
  \item A problem if user rights have been changed - TOCTTOU
\end{itemize}

\pagebreak
\section*{Reference section} \label{sec:reference}
\begin{description}
	\item[placeholder] \hfill \\
\end{description}
\end{document}
