\documentclass{article}

\usepackage[%
    left=0.5in,%
    right=0.5in,%
    top=0.5in,%
    bottom=0.5in,%
]{geometry}%
\usepackage{minitoc}
\usepackage{multicol}
\usepackage{graphicx}
\usepackage{fixltx2e}
\usepackage{listings}
\usepackage{color}
\usepackage{hyperref}
    \hypersetup{ colorlinks = true, linkcolor = blue }
\usepackage{blindtext}
\definecolor{lightgray}{gray}{0.9}
\graphicspath{ {./} }

\newcommand{\inlinecode}[2]{\colorbox{lightgray}{\lstinline
[language=#1]$#2$}}
\newcommand{\worddef}[1]{\hyperref[sec:reference]{\textit{#1}}}

\begin{document}

\tableofcontents

\newpage

\section{Role of the OS}
\begin{itemize}
  \item Compactly combine: 
  \begin{itemize}
    \item Identification 
    \item Authentication 
    \item Access control 
    \item Auditing
  \end{itemize} 
  \item User accounts to store permissions 
  \item Installation and configuration
\end{itemize}

\section{Authentication and Authorisation}
\begin{itemize}
  \item Principal
  \item Operation
  \item Reference Monitor
  \item Object
\end{itemize}

\subsection{Principle vs subject}
\begin{flushleft}
Principle: An entity that can be \textbf{granted access to objects} or can make statements affecting access control decisions. E.g user identity in an OS, used when discussing security policies
\end{flushleft}
\begin{flushleft}
Subject An active entity within an IT system. E.g. \textbf{process running under a user identity}, used when discussing operational systems enforcing policies.
\end{flushleft}

\subsection{Objects}
\begin{itemize}
  \item Object – Files or resources 
  \item E.g. Memory, printers, directories 
  \item Two options for focusing control: 
  \begin{itemize}
    \item What a subject is allowed to do. E.g. Word is allowed to write to a file
    \item What may be done to an object. E.g. An object cannot be deleted, but can be executed.
  \end{itemize}
  \item The access control is usually a compromise beween these two.
  \item In Unix, \textbf{everything is a file} 
  \item Files really represent resources 
  \item Organised in a tree structure, with alterations depending on the file system 
  \item Inodes store permission information 
  \item Every resource has an owner and a group
  \item Every resource has permission bits - held in the inode metadata 
  \item Permissions for the user / group / others
\end{itemize}

\subsubsection{inodes}
\begin{itemize}
  \item Inodes in Unix and Linux store the metadata for files 
  \item Each file name links to an inode, which stores security information
  \item Directory permissions are slightly different to files: 
  \item r – List files within the directory 
  \item w – add or remove files 
  \item x – traverse directory, open files in the directory
\end{itemize}

\section{General model}
\begin{itemize}
  \item We’ll settle on some common access on files: 
  \item Read – Simply viewing (Confidentiality) 
  \item Write – Includes changing, appending, deleting (Integrity) 
  \item Execute – Can run a file without knowing its contents
\end{itemize}

\section{Ownership}
\begin{itemize}
  \item \textbf{Discretionary}: Owner can be defined \textbf{for each resource}. Owner controls who gets access.
  \item \textbf{Mandatory}: There could be a system-wide policy 
  \item Most OS’s support the concept of ownership
  \item Even owner can't pass out a entity, if the receiver doesn't have permission for the file.
\end{itemize}

\section{Unix}
\begin{itemize}
  \item Unix simplifies access control by considering only the user, group and others 
  \item User is the current owner 
  \item Group is a named group entity 
  \item Everyone else 
  \item Unix offers Read, Write and Execute access controls
\end{itemize}

\subsection{Groups}
\begin{itemize}
  \item Users with similar access rights can be collected into groups 
  \item Groups are given permissions to access objects
\end{itemize}

\subsection{UID/GID}
\begin{itemize}
  \item Usernames in Unix / Linux are soft aliases, your UID is what determines permissions 
  \item User identities: UID 
  \item Group identities: GID 
  \item Your IDs are stored in /etc/passwd 
  \item Root has a special UID: 0
\end{itemize}

\section{The Shadow File}
\begin{itemize}
  \item In an attempt to improve password security, we can store password hashes in a shadow file 
  \item Readable only by root users 
  \item /etc/shadow stores the hashed passwords needed to authenticate users
\end{itemize}

\section{SUID}
\begin{itemize}
  \item Set UID: set the effective user to be the file owner when executed 
  \item Necessary to allow non-privileged access to privileged actions e.g. passwords 
  \item Dangerous, but allows specific files to execute as sudo, without requiring it.
\end{itemize}



\pagebreak
\section*{Reference section} \label{sec:reference}
\begin{description}
	\item[placeholder] \hfill \\
\end{description}
\end{document}
