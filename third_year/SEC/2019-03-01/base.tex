\documentclass{article}

\usepackage[%
    left=0.5in,%
    right=0.5in,%
    top=0.5in,%
    bottom=0.5in,%
]{geometry}%
\usepackage{minitoc}
\usepackage{multicol}
\usepackage{graphicx}
\usepackage{fixltx2e}
\usepackage{listings}
\usepackage{color}
\usepackage{hyperref}
    \hypersetup{ colorlinks = true, linkcolor = blue }
\usepackage{blindtext}
\definecolor{lightgray}{gray}{0.9}
\graphicspath{ {./} }

\newcommand{\inlinecode}[2]{\colorbox{lightgray}{\lstinline
[language=#1]$#2$}}
\newcommand{\worddef}[1]{\hyperref[sec:reference]{\textit{#1}}}

\begin{document}

\tableofcontents

\newpage

\section{Windows}
\begin{itemize}
  \item Windows predominantly uses Access Control Lists, and has done since Windows NT 
  \item Extends the usual read, write and execute with: 
  \begin{itemize}
  \item Take ownership 
  \item Change permissions 
  \item Delete 
  \end{itemize}
  \item 32-bit access masks (cf. Unix 9-bit)
\end{itemize}

\subsection{Access Control Matrix}
\begin{flushleft}
In principle, it would be great to store permissions in the matrix, but it's not feasable as memory would not scale
\end{flushleft}

\subsection{Access Control List}
\begin{flushleft}
Stored with an object itself, corresponding to a column of an ACM. For each file we store permissions (same as in unix). \textbf{This makes it diffucult to estimate users abiltities on the system}, e.g. Which files in the system can Alice modify
\end{flushleft}

\section{Access Control}
\begin{itemize}
  \item Access control in windows treats more than just files, also: 
  \begin{itemize}
  \item Registry keys 
  \item Active directory objects 
  \item Groups 
  \end{itemize}
  \item Inheritance is implemented: file can inherit ACLs from \textbf{parent directories}
  \item This allows to set defaults like the owner of all files is the owner of AD
\end{itemize}

\section{Principals}
\begin{itemize}
  \item Principals more broadly defined as well: 
  \item Local users 
  \item Domain users 
  \item Groups 
  \item Machines 
  \item Each principal has a human-readable name and security ID (SID)
\end{itemize}

\section{Local / Domain Principals}
\begin{itemize}
  \item LSA creates local principals 
  \item principal = MACHINE/principal 
  \item Domain principals administered on DC by domain admins 
  \item principal@domain = DOMAIN/principal
  \item net user /domain 
  \item net group /domain
  \item net localgroup /domain
\end{itemize}

\subsection{Groups}
\begin{itemize}
  \item Groups are collections of SIDs (object-orientated) 
  \item Group can itself be an SID 
  \item Groups can thus be nested 
  \item Groups are not nest-able on local machines 
  \item Managed by a domain controller within Active Directory
\end{itemize}

\section{Objects}
\begin{itemize}
\item Objects are passive entities in access operations 
\begin{itemize}
  \item Owner SID
  \item Primary group
  \item DACL - discretionary access control list
  \item SACL - security access control list
\end{itemize}
\item In Windows: 
\begin{itemize}
\item Executive objects (processes, threads, etc.) 
\item Private objects (files, directories) 
\end{itemize}
\item Securable objects have a security descriptor 
\begin{itemize}
\item Built-in securable objects managed by the OS 
\item Private objects managed by application software
\end{itemize}
\end{itemize}

\subsection{Access Tokens}
\begin{itemize}
\item Security credentials for a login session stored in access token 
\item Identifies the user, the user’s groups, and the user’s privileges
\item Structure:
\begin{itemize}
  \item User SID
  \item Groups and Alias SID
  \item Privileges
  \item Defaults for New Objects
  \item Miscellaneous
\end{itemize}
\end{itemize}

\section{Subjects}
\begin{itemize}
  \item Windows subjects: Processes and threads 
  \item New processes get a copy of the parent access token, possibly modified 
  \item Individual access tokens are immutable, and can live beyond policy changes (TOCTTOU issue)
\end{itemize}

\subsection{User Account Control}
\begin{itemize}
  \item After Vista, administrator users do not use an administrative access token by default 
  \item Users have two tokens, one heavily restricted and used by default 
  \item A prompt allows a user to spawn a process with the other token, or switch a process’ token
\end{itemize}

\subsection{Domains}
\begin{itemize}
  \item Single sign-on for network resources 
  \item Centralised security administration 
  \item Domain Controller (DC): handles user accounts and access control, trusted 3rd party for authentication 
  \item Multiple DCs allow for decentralisation by design
\end{itemize}

\section{Kerberos}
\begin{itemize}
  \item Widely supported, in particular is the default authentication for network logon in Windows 
  \item Uses symmetric encryption 
  \item Requires a trusted third party
\end{itemize}

\subsection{Important features}
\begin{itemize}
  \item Including nonces / timestamps prevents replay attacks 
  \item But, clocks must be synchronised between principals 
  \item Windows Kerberos buries domain group IDs inside tickets, for access checks 
  \item The ticket granting ticket usually exists until log-off, or rotates daily 
  \item A problem if user rights have been changed - TOCTTOU
\end{itemize}


\pagebreak
\section*{Reference section} \label{sec:reference}
\begin{description}
	\item[placeholder] \hfill \\
\end{description}
\end{document}
