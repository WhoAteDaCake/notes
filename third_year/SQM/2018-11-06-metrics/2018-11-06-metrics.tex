\documentclass{article}

\usepackage[%
    left=0.5in,%
    right=0.5in,%
    top=0.5in,%
    bottom=0.5in,%
]{geometry}%
\usepackage{minitoc}
\usepackage{multicol}
\usepackage{graphicx}
\usepackage{fixltx2e}
\usepackage{listings}
\usepackage{color}
\usepackage{hyperref}
    \hypersetup{ colorlinks = true, linkcolor = blue }
\usepackage{blindtext}
\definecolor{lightgray}{gray}{0.9}
\graphicspath{ {./} }

\newcommand{\inlinecode}[2]{\colorbox{lightgray}{\lstinline
[language=#1]$#2$}}
\newcommand{\worddef}[1]{\hyperref[sec:reference]{\textit{#1}}}

\begin{document}

\tableofcontents

\newpage

\section{Why Measure Software?}

\begin{itemize}
  \item To determine the quality of the current product or process and to make informed comparisons
  \item To predict qualities of a product/process
  \item To improve quality of a product/process (often continuously – \worddef{kaizen})
\end{itemize}

\section{Metrics}

\subsection{Example metrics}
\begin{itemize}
  \item Error rates / Defect rates 
  \item A defect is a failure to do something 
  \item An error is a wrong result or illegal action 
  \item Code generation rates. Measured by: per \textbf{person}, per \textbf{team}, per \textbf{project}. Will involve \textbf{aggregation} – see later
  \item Errors should be categorized by origin, type etc.
\end{itemize}

\subsection{Main objectives of software quality metrics}

\begin{itemize}
  \item Facilitate management control, planning and managerial intervention. Based on deviations of actual 
  \begin{itemize}
    \item performance from planned.
    \item timetable and budget performance from planned. 
  \end{itemize}
  \item 2. Identify situations for development or maintenance process improvement (preventive or corrective actions). Based on:
  \begin{itemize}
    \item Accumulation of metrics information regarding the performance of teams, units, etc.
  \end{itemize}
\end{itemize}

\subsection{Software quality metrics — Requirements}

\begin{flushleft}
General	requirements:
\begin{itemize}
  \item Relevant
  \item Valid
  \item Reliable
  \item Comprehensive 
  \item Mutually exclusive
\end{itemize}
Operative requirements:
\begin{itemize}
  \item Easy and simple 
  \item Does not require independent data collection 
  \item Immune to biased interventions by interested parties
\end{itemize}
\end{flushleft}

\subsection{Metrics evaluations}

\begin{flushleft}
Metrics can be evaluated on:
\begin{itemize}
  \item \textbf{Products}: Explicit results of software development activities. Deliverables, documentation, other artifacts produced
  \item \textbf{Process}: Activities	related	to	production	of	software
  \item \textbf{Resource}: Inputs into the software development activities, such as hardware, knowledge, people.
\end{itemize}
\end{flushleft}

\subsection{Process metrics}

\begin{flushleft}
Rates of production / productivity, quality of estimates, burn down rate (project velocity), refactoring rate. These can lead to \textbf{long term process improvement} by driving reflection on the process and suggestions for improvement
\end{flushleft}

\subsection{Product metrics}

\begin{flushleft}
Error densities, complexity measures, code quality measures, speed and memory measures, code profiles etc, track potential risks, uncover problem areas, Adjust workflow or tasks. They \textbf{evaluate teams ability to control quality}
\end{flushleft}

\subsection{Types of measures}
\begin{itemize}
  \item Direct: Cost, effort, LOC, speed, memory
  \item Indirect: functionality, quality, efficiency, reliablity, maintainability 
\end{itemize}


\subsection{Size oriented metrics}
\begin{itemize}
  \item Size of software produced
  \item LOC - Lines of Code or KLOC - 1000 Lines of code
  \item SLOC statement lines of code (no whitespace)
  \item Number of function points (NFP)
  \item Open used as part of density CONT
\end{itemize}

\subsection{Process metrics categories}
\begin{itemize}
  \item Software oricess quality metrics: Error density, severity metrics
  \item Software process timetable, error removal effectiveness and productivity metrics.
\end{itemize}

\section{Process metrics}


\subsection{Definitions}

\begin{flushleft}
  \item NCE = total detected by code inspections and testing.
  \item NDE = total errors detected in the development process.
  \item WCE = weighted NCE
  \item WDE = weighted NDE
\end{flushleft}


\pagebreak
\section*{Reference section} \label{sec:reference}
\begin{description}
	\item[measure] \hfill \\ A quantitative indication of extent, amount, dimension, capacity, or size of some attribute of a product or process. for example: Lines of code (LOC), Number of errors
	\item[metric] \hfill \\ quantitative measure of degree to which a system, component or process possesses a given attribute. That is to say a “derived measure.” May involve “guesstimates” where human processes are concerned. for example: Error density.
	\item[kaizen] \hfill \\ An approach to \textbf{continuous} organization improvement with emphasis on continuous measure and understanding of process.
\end{description}
\end{document}
