\documentclass{article}

\usepackage[%
    left=0.5in,%
    right=0.5in,%
    top=0.5in,%
    bottom=0.5in,%
]{geometry}%
\usepackage{minitoc}
\usepackage{multicol}
\usepackage{graphicx}
\usepackage{fixltx2e}
\usepackage{listings}
\usepackage{color}
\usepackage{hyperref}
    \hypersetup{ colorlinks = true, linkcolor = blue }
\usepackage{blindtext}
\definecolor{lightgray}{gray}{0.9}
\graphicspath{ {./} }

\newcommand{\inlinecode}[2]{\colorbox{lightgray}{\lstinline
[language=#1]$#2$}}
\newcommand{\worddef}[1]{\hyperref[sec:reference]{\textit{#1}}}

\begin{document}

\tableofcontents

\newpage

\section{A sample of possible causes for defects}

\begin{itemize}
  \item Incomplete or erroneous specification
  \item Missinterpretation of customer communication
  \item Intentional deviation from specifications
  \item Violation of programming standards
  \item Errors in data repressentation
  \item Inconsistent component interface
  \item Errors in design logic
  \item Incomplete or erroneous testing
  \item Inaccurate or incompletete documentation
  \item Errors in programming language design
  \item Ambigious or inconsistent human/computer interface.
\end{itemize}

\subsection{Six sigma}

\begin{itemize}
  \item The most wideky used strategy for statistical quality assurance
  \item Uses data and statistical analysis to measure and improve a company's operational performance
  \item Identifies and eliminates defects in manufacturing and service related process
  \item The "Six sigma" refers to six standard deviations.
\end{itemize}

\subsection{About}

\begin{flushleft}
The core steps:
\begin{itemize}
  \item Degine customer requirements, deliverables, and project goals via well-defined methods of customer communication
  \item Measure the existing process and its output to determine current quality performance (collect detect metrics)
  \item Analyze defect metrics and determine the vital  few causes (the 20\%)
\end{itemize}
Additional steps:
\begin{itemize}
  \item Improve the process by eliminating the root causes of defects
  \item Control the process to ensure that future work does not reintroduce the casues of defects.
\end{itemize}
\end{flushleft}

\section{Software reliability, availability and safety}

\subsection{Mean time to failure (MTTF)}

\begin{itemize}
  \item The time that a system is \textbf{not failed}, or available
  \item Often reffered to as updtime - the length of time that a system is online between outages or failures can be though of as the \textbf{time to failure} for that system
  \item Most systems only occasionally fail, so it is important to think of reliability in statistical terms. Manufacturers often run 
  \item CONT 
\end{itemize}

\subsection{Mean time to repair (MTTR)}
\begin{itemize}
  \item The amount of time required to repair a system and bring it back online is the \textbf{time to repair}.
  \item It can often take quite a while to diagnose, replace, or repair the failure. Even so, MTTR in IT systems tends to be measured in hours rather than days.
\end{itemize}

\pagebreak
\section*{Reference section} \label{sec:reference}
\begin{description}
	\item[placeholder] \hfill \\
\end{description}
\end{document}
