\documentclass{article}

\usepackage[%
    left=0.5in,%
    right=0.5in,%
    top=0.5in,%
    bottom=0.5in,%
]{geometry}%
\usepackage{minitoc}
\usepackage{multicol}
\usepackage{graphicx}
\usepackage{fixltx2e}
\usepackage{listings}
\usepackage{color}
\usepackage{hyperref}
    \hypersetup{ colorlinks = true, linkcolor = blue }
\usepackage{blindtext}
\definecolor{lightgray}{gray}{0.9}
\graphicspath{ {./} }

\newcommand{\inlinecode}[2]{\colorbox{lightgray}{\lstinline
[language=#1]$#2$}}
\newcommand{\worddef}[1]{\hyperref[sec:reference]{\textit{#1}}}

\begin{document}

\tableofcontents

\newpage

\section{Shift reduce parcing}

\subsection{Concept}
\begin{flushleft}
Construct a DFA where each state is labelled by “all possibilities” given the input and reductions thus far. (Similar to how an NFA is turned into a DFA.). Whenever reduction is possible, if there \textbf{is only one} possible reduction, then it is always clear what to do.
\end{flushleft}

\subsection{LL, LR and LALR parsing}
LL(k)
\begin{itemize}
  \item Input scanned \textbf{L}eft to right
  \item \textbf{L}eft most derivation
  \item \textbf{k} symbols of lookahead
\end{itemize}
LR(k)
\begin{itemize}
  \item Input scanned \textbf{L}eft to right
  \item \textbf{R}ightmost derivation in reverse
  \item \textbf{k} symbols of lookahead
\end{itemize}
LALR(k)
\begin{itemize}
  \item \textbf{L}ook\textbf{A}head \textbf{LR} (simplified LR parsing)
\end{itemize}

\subsection{Why}
\begin{flushleft}
These methods handle a wide class of grammars of practical significance. In particular, handles left- and right-recursive grammars (but left rec. needs less stack). LALR is a good compromise between expressiveness and space cost of implementation. Consequently, many parser generator tools based on LALR.
\end{flushleft}

\subsection{Terminology}
\begin{itemize}
  \item An item for a CFG is a production with a dot anywhere in the RHS.
\end{itemize}


\pagebreak
\section*{Reference section} \label{sec:reference}
\begin{description}
	\item[placeholder] \hfill \\
\end{description}
\end{document}
