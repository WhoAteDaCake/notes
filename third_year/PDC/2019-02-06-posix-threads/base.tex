\documentclass{article}

\usepackage[%
    left=0.5in,%
    right=0.5in,%
    top=0.5in,%
    bottom=0.5in,%
]{geometry}%
\usepackage{minitoc}
\usepackage{multicol}
\usepackage{graphicx}
\usepackage{fixltx2e}
\usepackage{listings}
\usepackage{color}
\usepackage{hyperref}
    \hypersetup{ colorlinks = true, linkcolor = blue }
\usepackage{blindtext}
\definecolor{lightgray}{gray}{0.9}
\graphicspath{ {./} }

\newcommand{\inlinecode}[2]{\colorbox{lightgray}{\lstinline
[language=#1]$#2$}}
\newcommand{\worddef}[1]{\hyperref[sec:reference]{\textit{#1}}}

\begin{document}

\tableofcontents

\newpage

\section{Introduction to Threads}
\begin{itemize}
  \item A thread is a software abstraction of a single processor
  \item Some threads (“system” threads) are \textbf{supported by the operating system}, which schedules those threads over the \textbf{physical processors}
  \item Other types of threads (“user space” or “green” threads) use only cooperative multithreading and all share (at most) a \textbf{single physical processor}. Typically more efficient, but not parallel.
  \item There can be \textbf{more threads than physical processors}. Threads are \textbf{allocated chunks of time} (time-slicing) on the physical processor(s), and with system threads are pre-empted if they run for longer to allow another thread to run
\end{itemize}

\section{The Locality Rule}
\begin{flushleft}
\textbf{Locality Rule}: fast programs tend to maximize the number of local memory references and minimize the number of non-local memory references
\end{flushleft}

\section{Limits to Performance Gain}

\subsection{Contention}
\begin{itemize}
  \item The main memory bus has limited bandwidth 
  \item It is a shared resource \textbf{used by all cores}, which compete to transfer data (one at a time) 
  \item At some point getting values from (and/or to) main memory will becomes \textbf{the limiting factor}, at least for data-intensive applications
\end{itemize}

\pagebreak
\section*{Reference section} \label{sec:reference}
\begin{description}
	\item[placeholder] \hfill \\
\end{description}
\end{document}
