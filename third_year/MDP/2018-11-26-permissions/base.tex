\documentclass{article}

\usepackage[%
    left=0.5in,%
    right=0.5in,%
    top=0.5in,%
    bottom=0.5in,%
]{geometry}%
\usepackage{minitoc}
\usepackage{multicol}
\usepackage{graphicx}
\usepackage{fixltx2e}
\usepackage{listings}
\usepackage{color}
\usepackage{hyperref}
    \hypersetup{ colorlinks = true, linkcolor = blue }
\usepackage{blindtext}
\definecolor{lightgray}{gray}{0.9}
\graphicspath{ {./} }

\newcommand{\inlinecode}[2]{\colorbox{lightgray}{\lstinline
[language=#1]$#2$}}
\newcommand{\worddef}[1]{\hyperref[sec:reference]{\textit{#1}}}

\begin{document}

\tableofcontents

\newpage

\section{Permissions}

\subsection{Old	Permissions}

\begin{flushleft}
Show permissions required at install time. Not prompted again regarding permissions at run-time.
\begin{itemize}
  \item Not yet made a commitment (financial, mental) to the application. Can compare to other applications 
  \item \textbf{Not per session / at run-time }
  \item “Seamless” switching between Activities / applications 
  \item Would slow down the user experience 
  \item Train users to click “ok” repeatedly without considering the implications
\end{itemize}
\end{flushleft}

\subsection{New permissions}

\begin{itemize}
  \item No access by default. Control access to specific mechanisms
  \item Applications can offer protected access to resources and data with permissions. Permissions explicitly granted by users / the system
  \item Permission architecture
  \begin{itemize}
    \item Applications statically declare permissions 
    \item Required of components interacting with them. \textbf{You must} have this permission to interact with me 
    \item Required by components they interact with. \textbf{I will} need these permissions 
    \item Android requires user’s consent for specific permissions
  \end{itemize}
\end{itemize}

\section{Normal or Dangerous}
\begin{flushleft}
Normal
\begin{itemize}
  \item Do not directly risk the user’s privacy 
  \item Network state, Internet, Alarms, Wallpaper… 
  \item Granted automatically 
  \item But still must be declared in the manifest
\end{itemize}
Dangerous
\begin{itemize}
  \item Potentially do risk the user’s privacy 
  \item The user must explicitly approve the permission request 
  \item Need to think what to do if permission is denied
\end{itemize}
Some permissions:
\begin{itemize}
  \item Cost-Sensitive APIs: Telephony, SMS/MMS, In-App Billing, NFC Access
  \item Personal Information. Contacts, calendar, messages, emails 
  \item Device Meta-data . System logs, browser history, network identifiers 
  \item Sensitive Input Devices. Interaction with the surrounding environment: Camera, microphone, GPS
\end{itemize}
\end{flushleft}

\newpage

\section{Component Permissions}

\begin{flushleft}
Activity
\begin{itemize}
  \item Restricts which components can start the activity 
  \item Checked within execution of: \verb|startActivity(), startActivityForResult()|
\end{itemize}
Service
\begin{itemize}
  \item Restricts which components can start or bind to the associated service 
  \item Checked within execution of: \verb|Context.startService(), Context.stopService(), Context.bindService()|
\end{itemize}
Others
\begin{itemize}
  \item ContentProvider: Restricts which components can read or write to a ContentProvider
  \item BroadcastReceiver: Restricts which components can register to receive a certain Broadcast 
  \item Throw SecurityException on permissions failure: Usually as we’ve forgotten to ask for permission during installation
\end{itemize}
\end{flushleft}

\section{Runtime Permissions}

\begin{flushleft}
Each time we want to do something “dangerous”
\begin{itemize}
  \item \verb|shouldShowRequestPermissionRationale(…)| True if the user has previously denied the request
  \item \verb|requestPermissions(…)|
  \item \verb|onRequestPermissionsResult(…)| Either do the dangerous thing, or gracefully degrade the functionality of the app
\end{itemize}
\end{flushleft}

\section{Permissions vs Use}

\begin{itemize}
  \item Read your text messages: To confirm your phone number via text message (if you’ve added it to your account)
  \item Read/write contacts: To import and sync your phone’s contacts to Facebook, or vice versa (think updating contact images)
  \item Add and/or modify calendar events and send emails to guests without your knowledge: To see your Facebook events in your phone’s calendar
  \item Read calendar events plus confidential information: To check your calendar for you to see if you have something already scheduled for the time of the Facebook event you’re currently viewing 
  \item Good practice to explain why an application needs a permission, especially if now not having it will prevent it from functioning
\end{itemize}

\section{Temporary URI Permissions}

\begin{flushleft}
\textbf{Applications making use of multiple Activities}.
“Access	to	the	mail	should	be	protected	by	permissions,	
since	this	is	sensitive	user	data.	However,	if	a	URI	to	an	
image	attachment	is	given	to	an	image	viewer,	that	image	
viewer	will	not	have	permission	to	open	the	attachment	
since	it	has	no	reason	to	hold	a	permission	to	access	all	email.
\textbf{Allow access to specific URIs, not the whole provider}.\\
\end{flushleft}
\begin{flushleft}
Temporary URI permissions last while the stack of the receiving Activity is active
\end{flushleft}


\pagebreak
\section*{Reference section} \label{sec:reference}
\begin{description}
	\item[placeholder] \hfill \\
\end{description}
\end{document}
