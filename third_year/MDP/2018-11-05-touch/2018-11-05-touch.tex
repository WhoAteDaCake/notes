\documentclass{article}

\usepackage[%
    left=0.5in,%
    right=0.5in,%
    top=0.5in,%
    bottom=0.5in,%
]{geometry}%
\usepackage{minitoc}
\usepackage{multicol}
\usepackage{graphicx}
\usepackage{fixltx2e}
\usepackage{listings}
\usepackage{color}
\usepackage{hyperref}
    \hypersetup{ colorlinks = true, linkcolor = blue }
\usepackage{blindtext}
\definecolor{lightgray}{gray}{0.9}
\graphicspath{ {./} }

\newcommand{\inlinecode}[2]{\colorbox{lightgray}{\lstinline
[language=#1]$#2$}}
\newcommand{\worddef}[1]{\hyperref[sec:reference]{\textit{#1}}}

\begin{document}

\section{Resistive Touch Screen}

\begin{itemize}
  \item Two sheets of transparent layer with metallic (resistive) coating facing each other
  \item Four wires connection
  \item Excite alternate axes with voltage
  \item Touch acts as voltage divider for coordinate estimation
\end{itemize}

\subsection{Properties}

\begin{itemize}
  \item Used with finger or any pointing device (passive)
  \item Requires certain amount of pressure.
  \item Difficult, but can be altered for multi-touch (only two-points touch)
\end{itemize}

\subsection{Pros}
\begin{itemize}
  \item Stylus flexibility, can be touched with anything as it calculates pressure
  \item Higher sensor resolutions. Meaning a smaller tip will work on a screen
  \item Detects less accidental touches. For example if water is on the screen. For something like ATM machines, we don't want to be sensitive
  \item Cheaper
\end{itemize}

\subsection{Cons}
\begin{itemize}
  \item Very difficult to add accurate multitouch
  \item Low sensitivity can cause fustration from users
\end{itemize}


\section{Capacitive Touch Screen}

\begin{itemize}  
  \item Screen is covered in a capacitive material: Indium-tin-oxide (conductive, optically 90\% transparent)
  \item Capacitance = ability to store electrical charge (Works through glass)
  \item Human beings act as small capacitors
  \item Touching the screen modifies its electrostatic field
\end{itemize}

\subsection{Surface Capacitance}

\begin{itemize}
  \item Cover the screen with a uniform conductive material
  \item Apply a small voltage to generate an electrostatic field. Touching with a finger creates a dynamic capacitor
  \item Measure effective capacitance at each corner of the screen
  \begin{itemize}
    \item The larger the change, the closer to the corner the touch is
    \item Combine measurements from all corners = location of the touch
  \end{itemize}
  \item Will not work with multi-touch as clicking two points would confuse the field
\end{itemize}

\section{Projected (Mutual) Capacitance}

\begin{itemize}
  \item Conductive material is etched with rows / columns
  \item An electronic field is projected through the top layer of glass
  \item An electrostatic field is created
  \item Human acts as a conductor
  \item Decrease of capacitance between electrodes is detected during touch 
  \item Measure capacitance at each electrode grid point
\end{itemize}

\subsection{Properties}
\begin{itemize}
  \item More accurate / multi-touch
  \item Sensing requires an “active” touch: Non-conductive materials will not change the electrostatic field: fingers, capacitive glove, capacitive stylus
  \item State-of-art technique for smart phone, tablets.
  \item Can be manufactured to the top glass layer (on-cell) or the display layer (in-cell)
\end{itemize}

\section{Touch and UI}

\begin{itemize}
  \item Touch relies on finger contact with the display. This has to alter the way we design our displays
  \item Size of the finger sets the properties of the UI, not the size of a display
\end{itemize}

\pagebreak
\section*{Reference section} \label{sec:reference}
\begin{description}
	\item[placeholder] \hfill \\
\end{description}
\end{document}
