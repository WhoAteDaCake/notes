\documentclass{article}
\begin{document}

\begin{flushleft}
\section{Experimental Studies}
General patern:
\begin{itemize}
	\item Write a program implementing the algorithm
	\item Run the program with inputs of "varying size and composition"
	\item Use a system method to get an (in)accurate measure of the actual running time
	\item Plot the results
	\item Intterpret \& analyse
\end{itemize}
\end{flushleft}

\begin{flushleft}
\section{Limitations of Experiments}
\begin{itemize}
	\item It is necessary to implement the algorithm, which may be difficult or time consuming
	\item Results may not be indicative of the running time on other inputs not included in the experiment
	\item In order to compare two agorithms directly same hardware and software environments must be used
\end{itemize}
\end{flushleft}

\begin{flushleft}
\section{Limitations of Theory}
\begin{itemize}
	\item It is necessary to implement the theory, which may be difficult or time consuming
	\item Results may not be indicative of the typical running time on inputs encountered in the real world
\end{itemize}
\end{flushleft}

\begin{flushleft}
\section{Theoretical Ananlysis}
\begin{itemize}
	\item Uses a high-level description of the algorithm instead of implementation
	\item \textbf{Characterizes running time as a function of the input size, n}
	\item Takes into account all possible inputs
	\item Allows us to evaluate the speed of an algorithm intedependently of the hardware/software environment
\end{itemize}
\end{flushleft}

\end{document}