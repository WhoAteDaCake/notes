\documentclass{article}
\usepackage[%
    left=0.5in,%
    right=0.5in,%
    top=0.5in,%
    bottom=0.5in,%
]{geometry}%
\usepackage{minitoc}
\usepackage{multicol}
\usepackage{graphicx}
\usepackage{fixltx2e}
\usepackage{hyperref}
\usepackage{hyperref}
    \hypersetup{ colorlinks = true, linkcolor = blue }
\usepackage{blindtext}

\graphicspath{ {./} }

\newcommand{\worddef}[1]{\hyperref[sec:reference]{\textit{#1}}}

\begin{document}

\section{What is software maintenance}
\begin{flushleft}
	Traditionally, it means changing software \textbf{after} it has been delivered and is in use.\\
It includes:
\begin{itemize}
	\item Fixing code errors
	\item Fixing design problems
	\item Adding additional requirements
\end{itemize}
Up to \textbf{80\%} of software development can be considered maintenance.
\end{flushleft}

\section{Types of maintenance}
\begin{flushleft}
\begin{itemize}
	\item Corrective Maintenance
	\begin{itemize}
		\item Fixing errors in the system
		\item E.g. bugs
	\end{itemize}
	\item Adaptive Maintenance
	\begin{itemize}
		\item Modifying due to changes in the environment of the software
		\item E.g. new laws, change of business needs
	\end{itemize}
	\item Perfective/performance Maintenance
	\begin{itemize}
		\item Improving system without changing it's functionality
		\item E.g. improving performance, ease of maintenance
	\end{itemize}
\end{itemize}
\end{flushleft}

\section{Requirements for maintenance}
\begin{itemize}
	\item Understanting the client
	\item Understanding the code
	\item refactoring code
	\item Extending the code
	\item Working as a team
	\item Managing client expectations
	\item Managing maintenance process
\end{itemize}

\section{Maintenance}

\subsection{Class diagram}
\begin{flushleft}
	Show a set of classes, interfaces and collaborations, and their relationships. Addresses static design view of the system.\\
	\textbf{Classes:} blueprints for objects. Contain data/information and perform operations.
\end{flushleft}

\end{document}
