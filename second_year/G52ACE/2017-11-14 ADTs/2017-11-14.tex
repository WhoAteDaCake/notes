\documentclass{article}
\usepackage[%
    left=0.5in,%
    right=0.5in,%
    top=0.5in,%
    bottom=0.5in,%
]{geometry}%
\usepackage{minitoc}
\usepackage{multicol}
\usepackage{graphicx}
\usepackage{fixltx2e}
\usepackage{hyperref}
\usepackage{hyperref}
    \hypersetup{ colorlinks = true, linkcolor = blue }
\usepackage{blindtext}

\graphicspath{ {./} }

\newcommand{\inlinecode}[2]{\colorbox{lightgray}{\lstinline
[language=#1]$#2$}}
\newcommand{\worddef}[1]{\hyperref[sec:reference]{\textit{#1}}}

\begin{document}
\section{Abstract Data Types (ADTs)}
\begin{flushleft}
An abstract data type (ADT) is an abstraction of a data structure
An ADT specifies:
\begin{itemize}
	\item Data stored
	\item Operations on the data
	\item Error conditions associated with operations
\end{itemize}
An ADT does not \textbf{specify} the implementation \textbf{itself} - hence “abstract”
\end{flushleft}

\section{Concrete Data Types (CDTs)}
\begin{flushleft}
The actual date structure that we use:
\begin{itemize}
	\item Possibly consists of Arrays or similar data
	\item An ADT might be implemented using different choices for the CDT
	\item The choice of CDT will not be apparent from the interface: “data hiding” “encapsulation” – e.g. see ‘Object Oriented Methods’
	\item The choice of CDT will affect the runtime and space usage – and so is a major topic of this module
\end{itemize}
\end{flushleft}

\pagebreak
\section*{Reference section} \label{sec:reference}
\begin{description}
	\item[placeholder] \hfill \\
\end{description}
\end{document}
