\documentclass{article}
\usepackage[%
    left=0.5in,%
    right=0.5in,%
    top=0.5in,%
    bottom=0.5in,%
]{geometry}%
\usepackage{minitoc}
\usepackage{multicol}
\usepackage{graphicx}
\usepackage{fixltx2e}
\usepackage{hyperref}
\usepackage{hyperref}
    \hypersetup{ colorlinks = true, linkcolor = blue }
\usepackage{blindtext}

\graphicspath{ {./} }

\newcommand{\inlinecode}[2]{\colorbox{lightgray}{\lstinline
[language=#1]$#2$}}
\newcommand{\worddef}[1]{\hyperref[sec:reference]{\textit{#1}}}

\begin{document}

\section{Cloud computing}
\subsection{What is cloud computing}
\begin{flushleft}
A new computational \textbf{paradigm}. \textbf{Out-source} your computation and storage needs to a well-managed data center. No worries about the physical machines: power, cooling, maintenance. Virtualisation provides the necessary \textbf{isolation} to share multiple clients on a single physical machine
\end{flushleft}

\subsection{Characteristics}
\begin{flushleft}
A number of characteristics define cloud data, applications services and infrastructure: 
\begin{itemize}
	\item Remotely hosted: services or data are hosted on remote infrastructure
	\item No-need-to-know in terms of the underlying details of infrastructure, application interface with the infrastructure via the APIs
	\item Ubiquitous: services or data are available from anywhere (\textbf{always on, anywhere and any place})
	\item Flexibility and elasticity allow these systems to scale up and down at will
	\item Utilising resources of all kinds CPU, storage, server capacity, load balancing, and databases
	\item Commodified: the result is a utility computing model similar to traditional utilities, like gas and electricity - “\textbf{pay as much as used and needed}" !
\end{itemize}
\end{flushleft}

\subsection{Rapid elasticity}
\begin{flushleft}
\textbf{Rapid elasticity} is currently one of the key challenges in Cloud Computing. Capabilities can be rapidly and elastically provisioned, in some cases \textbf{automatically}, to quickly scale out and rapidly released to scale in. To the consumer, resources often appear to be \textbf{unlimited} and can be purchased in any quantity at any time. Optimisation of the current usage will save energy! (e.g. putting unused machines to sleep). \textbf{Prediction techniques} may help alleviate the problem based on \textbf{historical data} (e.g. machine learning applied to cloud computing).
\end{flushleft}

\subsection{Migration}
\begin{flushleft}
Non-live Migration:
\begin{itemize}
	\item  Instead of shutting down the computer, we could pause the VM (\textbf{checkpointing})
	\item Then, copy over the memory pages used by the VM to the new hardware as quickly as possible meaning less downtime, but still noticeable 
\end{itemize}
Live Migration:
\begin{itemize}
	\item the idea is to start moving the virtual machine while it is still \textbf{operational} (pre-copy memory migration)
\end{itemize}
\end{flushleft}

\subsection{Cloud service models}
\begin{flushleft}
Three main service models:
\begin{itemize}
	\item Infrastructure as a service (Iaas)
	\item Platform as a service (Paas)
	\item Software as a service (Saas)
\end{itemize}
More recently, new services are being defined. For example, Desktop as a Service (\textbf{Daas})
\end{flushleft}

\subsubsection{Iaas}
\begin{itemize}
	\item Iaas provides resources of the \textbf{underlying} cloud infrastructure to customers
	\item Virtual machines (with different OSs) and other virtualised hardware, processing, storage, networks, etc 
	\item Example of Iaas: Amazon EC2 (Xen hypervisor - paravirtualisation)
	\item End-user: typically a system administrator
\end{itemize}

\subsubsection{Paas}
\begin{itemize}
	\item Paas provides service in the form of a platform on which the customer’s applications can run
	\item Tools to create your own applications (e.g. development environment, programming language tools)
	\item Examples: AppEngine, Microsoft Azure, Force.com, Heroku
	\item End-user: developers
\end{itemize}

\subsubsection{Saas}
\begin{itemize}
	\item Saas provides service to customer in the form of software
	\item Applications that run on the Cloud
	\item Examples: G-mail, Microsoft 365, Dropbox
	\item End-user: regular users
\end{itemize}

\pagebreak
\section*{Reference section} \label{sec:reference}
\begin{description}
	\item[placeholder] \hfill \\
\end{description}
\end{document}
